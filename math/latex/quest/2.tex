% {
% 	\section{Общий интеграл ОДУ 1-го порядка.}

% 	\subsection*{Определение 2.1}

% 	Рассмотрим ДУ 1-го порядка $F(x, y, y') = 0$ (2), где $F(t_1, t_2, t_3),\ \dfrac{\partial F}{\partial t_2},\ \dfrac{\partial F}{\partial t_3}$, непрерывны в области $\Omega \subset \mathbb{R}^3$.


% 	Общим интегралом уравнения (2) называется равенство $\Phi(x, y, c) = 0$ (3), где $\Phi(t_1, t_2, t_3)$ непрерывно дифференцируемо в некоторой области $G \subset \mathbb{R}_3$ и обладает свойством: если $y(x)$ непрерывно, дифференцировать равенство (3) по $x$, то

% 	(при этом получим $\dfrac{\partial \Phi}{\partial x} + \dfrac{\partial \Phi}{\partial y} \cdot y'(x) = 0$ (4)),  и исключить   сиз уравнений   (3)   и   (4) ,  то получим   ДУ,  эквивалентное исходному уравнению   (2)

% 	Уравнение   (2, общий интеграл)   называют также дифференциальным уравнением функций, заданных (возможно, неявно) уравнением   (3).

% 	\subsection*{Замечание}

% 	Не всегда общий интеграл (3) содержит все решения ДУ (2).

% 	\subsection*{Теорема 2.1}

% 	Добавим в определение 2.1 требование, чтобы равенство (3) было разрешимо относительно параметра $c$, то есть, имело вид


% 	\[
% 	c = \varphi(x, y) \tag{5}
% 	\]


% 	($\varphi$ — непрерывно дифференцируемая в некоторой области $\tilde{G}$)

% 	Тогда, если $y(x)$, определенная на $(a, b)$, и $(x, y(x)) \in \tilde{G}$, $(x, y, y')\in \Omega\, \forall x \in (a, b)$ — непрерывно дифференцируемая функция, удовлетворяющая уравнению (2), при некотором значении параметра $c$, то она является решением ДУ (2).

% 	И обратно, любое решение ДУ (2), определенное на $(a, b)$, удовлетворяет равенству (5) при некотором значении $c$ ( то есть, в общем интеграле   (5)   содержатся все решения  ДУ ).

% 	\subsection*{Определение 2.2}

% 	ДУ 1-го порядка, разрешенное относительно производной, называется нормальным:


% 	\[
% 	y' = f(x, y) \tag{6}
% 	\]



% 	\subsection*{Теорема 2.2}

% 	Существование и единственность решения задачи Коши для нормального

% 	ДУ 1-го порядка (6).

% 	Пусть $f,\ f_y$ непрерывны в области $G \subset \mathbb{R}^2$. Тогда $\forall(x_0, y_0) \in G\ \exists!$ решение
% 	$y(x)$ уравнения (6), определенное на некотором промежутке $\left[x_0 - h,\ x_0 + h \right]$ и удовлетворяющее начальному условию $y(x_0) = y_0$ (  hсвое для каждой точки ) ($x_0,\ y_0$ — любая точка $G$).

% 	То есть, через каждую точку $G$ проходит ровно одна интегральная кривая.

% 	(без доказательства)

% 	\newpage
% }