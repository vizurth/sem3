{
	\section{Функция $cos(z), sin(z)$ дальше переписать из списка}

\subsection*{Определение 8.2.}

\medskip

1). Рассмотрим ряд
\[
\sum_{n=0}^{\infty} (-1)^n \frac{z^{2n}}{(2n)!}.
\]

Ряд сходится абсолютно $\ \forall z \in \mathbb{C}$, так как ряд
\[
\sum_{n=0}^{\infty} (-1)^n \frac{x^{2n}}{(2n)!}
\]
сходится абсолютно $\ \forall x \in \mathbb{R}$, и
\[
\forall z \in \mathbb{R} \quad
\sum_{n=0}^{\infty} (-1)^n \frac{z^{2n}}{(2n)!} = \cos z.
\]

Назовем сумму этого ряда $S(z)$ значением функции $\cos z$
$\ \forall z \in \mathbb{C}$

( мы продолжили функцию $\cos x$ с действительной оси
на всю комплексную плоскость ).

\medskip

2). Рассмотрим ряд
\[
\sum_{n=0}^{\infty} (-1)^n \frac{z^{2n+1}}{(2n+1)!}.
\]

Ряд сходится абсолютно $\ \forall z \in \mathbb{C}$, так как ряд
\[
\sum_{n=0}^{\infty} (-1)^n \frac{x^{2n+1}}{(2n+1)!}
\]
сходится абсолютно $\ \forall x \in \mathbb{R}$, и
\[
\forall z \in \mathbb{R} \quad
\sum_{n=0}^{\infty} (-1)^n \frac{z^{2n+1}}{(2n+1)!} = \sin z.
\]

Назовем сумму этого ряда $S(z)$ значением функции $\sin z$
$\ \forall z \in \mathbb{C}$

( мы продолжили функцию $\sin x$ с действительной оси
на всю комплексную плоскость ).

\subsection*{Замечание 8.1.}

Справедлива теорема: если существует продолжение бесконечно
дифференцируемой функции с действительной оси на комплексную
плоскость с сохранением свойства бесконечной дифференцируемости,
то такое продолжение единственно.
(Такое продолжение называется аналитическим).

\medskip

Построенные нами продолжения функций $e^{x}$, $\cos x$, $\sin x$
бесконечно дифференцируемы как суммы степенных рядов.
Следовательно, построенные нами продолжения — это единственно
возможные аналитические продолжения функций
$e^{x}$, $\cos x$, $\sin x$ с действительной оси на комплексную
плоскость).

\medskip

\subsection*{Замечание 8.2.}

\[
\text{Функция } \sin z =
\sum_{n=0}^{\infty} (-1)^n \frac{z^{2n+1}}{(2n+1)!}
\quad \text{сохранила свойство нечетности.}
\]

\[
\text{Функция } \cos z =
\sum_{n=0}^{\infty} (-1)^n \frac{z^{2n}}{(2n)!}
\quad \text{сохранила свойство четности.}
\]


\subsection*{Теорема 8.2.}

\[
\forall z \in \mathbb{C}
\]

1).
\[
e^{iz} = \cos z + i \sin z,
\]

2).
\[
\sin z = \frac{e^{iz} - e^{-iz}}{2i},
\]

3).
\[
\cos z = \frac{e^{iz} + e^{-iz}}{2}.
\]

\medskip

\subsection*{Следствия.}

\medskip

1).
\[
e^{x+iy} = e^{x} (\cos y + i \sin y),
\quad \forall x, y \in \mathbb{R}.
\]

\medskip

2).
\[
e^{z} \text{ имеет период } T = 2\pi i,
\]
так как
\[
e^{z+2\pi i} = e^{z} e^{2\pi i}
= e^{z} (\cos 2\pi + i \sin 2\pi) = e^{z}.
\]

\medskip

3).
\[
\cos z \text{ и } \sin z \text{ имеют период } 2\pi,
\]
так как
\[
\cos (z+2\pi) =
\frac{e^{iz+2\pi i} + e^{-iz-2\pi i}}{2}
=
\frac{e^{iz} + e^{-iz}}{2}
= \cos z;
\]

\[
\sin (z+2\pi) =
\frac{e^{iz+2\pi i} - e^{-iz-2\pi i}}{2i}
=
\frac{e^{iz} - e^{-iz}}{2i}
= \sin z.
\]

\medskip

4). Остаются в силе формула
\[
\sin^2 z + \cos^2 z = 1
\]
и другие тригонометрические формулы

\[
\left(
\sin^2 z + \cos^2 z =
\left( \frac{e^{iz} - e^{-iz}}{2i} \right)^2
+
\left( \frac{e^{iz} + e^{-iz}}{2} \right)^2
=
\right.
\]

\[
\left.
=
- \frac{e^{2iz} + e^{-2iz} - 2}{4}
+ \frac{e^{2iz} + e^{-2iz} + 2}{4}
= 1
\right)
\]

\medskip

5).
\[
\operatorname{sh}(iz) =
\frac{e^{iz} - e^{-iz}}{2}
= i \sin z,
\quad
\sin (iz) =
\frac{e^{-z} - e^{z}}{2i}
= i \operatorname{sh} z,
\]

\[
\operatorname{ch}(iz) =
\frac{e^{iz} + e^{-iz}}{2}
= \cos z,
\quad
\cos (iz) =
\frac{e^{-z} + e^{z}}{2}
= \operatorname{ch} z.
\]

\medskip

6). Исчезло свойство ограниченности функций
$\sin z$ и $\cos z$, например,

\[
\cos (ix) =
\frac{e^{-x} + e^{x}}{2}
\xrightarrow[x \to +\infty]{} +\infty,
\quad (x \in \mathbb{R}).
\]

\medskip

\subsection*{Определение 8.3.}

Пусть $z \in \mathbb{C} \setminus \{0\}$.
$w \in \mathbb{C}$ называется логарифмом $z$, если
\[
z = e^{w}.
\]

\medskip

\subsection*{Теорема 8.3.}

Пусть $z \in \mathbb{C} \setminus \{0\}$.
\[
\mathrm{Ln}\, z = \ln |z| + i \operatorname{Arg} z.
\]

\medskip

\subsection*{Определение 8.4.}

По определению
\[
z^{w} = e^{\,w \mathrm{Ln}\, z}.
\]

\medskip

\subsection*{Примеры.}

\medskip

1).
\[
\mathrm{Ln}(-1) = \ln 1 + i(\pi + 2\pi k)
= i\pi (2k+1),
\quad (k \in \mathbb{Z}).
\]

\medskip

2).
\[
(-1)^{\sqrt{2}} =
e^{\sqrt{2}\,\mathrm{Ln}(-1)}
=
e^{i\sqrt{2}\pi(2k+1)},
\quad (k \in \mathbb{Z}).
\]



	\newpage
}