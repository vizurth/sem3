{
	\section{Ряд Тейлора функции в точке. Пример: показать, что ряд Тейлора функции (посмотреть в списке и написать)}

	\subsection*{Определение 6.3} Пусть функция \( f(x) \) определена в некоторой окрестности \( U_{x_0} \) точки \( x_0 \) и имеет там производные всех порядков. Тогда степенной ряд \[ \sum_{n=0}^{\infty} \frac{f^{(n)}(x_0)}{n!} (x - x_0)^n \] называется \textit{рядом Тейлора} функции \( f(x) \) в точке \( x_0 \). Если \( x_0 = 0 \), то ряд называется \textit{рядом Маклорена}. 
	
	\subsection*{Замечание} Очевидно, что ряд Тейлора сходится к функции \( f(x) \) в точке \( x_0 \). Однако он может не сходиться к функции \( f(x) \) ни в какой другой точке.

	\subsection*{Пример} Рассмотрим функцию \[ f(x) = \begin{cases} e^{-1/x^2}, & x \neq 0 \\ 0, & x = 0 \end{cases} \] Покажем, что функция \( f(x) \) бесконечно дифференцируема в любой окрестности точки \( 0 \), однако её ряд Маклорена не сходится к ней ни в одной точке, кроме \( 0 \).

	\newpage
}