{
	\section{Общее решение однородной системы ЛДУ в случае, когда количество линейно-
	независимых собственных векторов матрицы системы совпадает с порядком системы.}

	\section*{Общее решение однородной системы ЛДУ при полном наборе собственных векторов}

Рассматривается однородная система линейных дифференциальных уравнений
\begin{equation}
Y'(x) = A\,Y(x),
\tag{7.3}
\end{equation}
где $A$ --- постоянная $n \times n$–матрица, $Y(x) \in \mathbb{R}^n$.

\subsection*{Напоминания: фундаментальная система, вронскиан}

\textbf{Замечание 7.1.}
\begin{enumerate}
  \item Множество вектор–функций образует линейное пространство.
  \item Остаются в силе определения линейной зависимости и независимости систем вектор–функций.
  \item Система вектор–функций $\{\Phi_k(x)\}_{k=1}^n$ называется \emph{фундаментальной системой решений} (ФСР) СЛДУ \eqref{7.3}, если все $\Phi_k(x)$ являются решениями \eqref{7.3}, их количество равно порядку системы $n$, и $\{\Phi_k(x)\}_{k=1}^n$ — линейно независимая система вектор–функций.
\end{enumerate}

\textbf{Определение 7.5.} Пусть


\[
\Phi_1(x) =
\begin{pmatrix}
\varphi_{11}(x) \\
\varphi_{21}(x) \\
\vdots \\
\varphi_{n1}(x)
\end{pmatrix},\quad
\Phi_2(x) =
\begin{pmatrix}
\varphi_{12}(x) \\
\varphi_{22}(x) \\
\vdots \\
\varphi_{n2}(x)
\end{pmatrix},\quad
\ldots,\quad
\Phi_n(x) =
\begin{pmatrix}
\varphi_{1n}(x) \\
\varphi_{2n}(x) \\
\vdots \\
\varphi_{nn}(x)
\end{pmatrix}
\]


— решения системы \eqref{7.3}. Тогда определитель


\[
W(x) =
\begin{vmatrix}
\varphi_{11}(x) & \varphi_{12}(x) & \ldots & \varphi_{1n}(x) \\
\varphi_{21}(x) & \varphi_{22}(x) & \ldots & \varphi_{2n}(x) \\
\vdots          & \vdots          & \ddots & \vdots          \\
\varphi_{n1}(x) & \varphi_{n2}(x) & \ldots & \varphi_{nn}(x)
\end{vmatrix}
\]


называется \emph{вронскианом} этого набора решений.

\textbf{Теорема 7.3.} Пусть $\{\Phi_k(x)\}_{k=1}^n$ — система решений СЛДУ \eqref{7.3}. Тогда равносильны утверждения:
\begin{enumerate}
  \item $\{\Phi_k(x)\}_{k=1}^n$ — ФСР системы \eqref{7.3};
  \item вронскиан $W(x)$ не равен тождественно нулю на $(a,b)$ (то есть $\exists\,x_0 \in (a,b): W(x_0) \neq 0$);
  \item $W(x) \neq 0$ для всех $x \in (a,b)$.
\end{enumerate}

Из теоремы о существовании и единственности (Теорема 7.2) следует, что общее решение однородной системы $Y'(x)=A(x)Y(x)$ содержит ровно $n$ произвольных постоянных.

\subsection*{Формула общего решения при полном наборе собственных векторов}

Пусть матрица $A$ имеет $n$ линейно независимых собственных векторов


\[
v_1,\ldots,v_n \in \mathbb{C}^n, \quad
A v_k = \lambda_k v_k, \quad k=1,\ldots,n.
\]


(Если все $\lambda_k$ вещественные и $v_k$ можно выбрать вещественными, работаем в $\mathbb{R}^n$; общий вид в $\mathbb{C}^n$ тот же, а вещественное решение строится стандартно через действительную и мнимую части.)

\textbf{Утверждение.} Тогда вектор–функции


\[
\Phi_k(x) = e^{\lambda_k x} v_k,\quad k=1,\ldots,n,
\]


образуют фундаментальную систему решений однородной системы \eqref{7.3}, а общее решение имеет вид


\[
Y(x) = c_1 e^{\lambda_1 x} v_1 + c_2 e^{\lambda_2 x} v_2 + \ldots + c_n e^{\lambda_n x} v_n,
\]


где $c_1,\ldots,c_n$ — произвольные постоянные.

\subsection*{Доказательство}

\textbf{Шаг 1. Каждая $\Phi_k(x)$ — решение.}

Рассмотрим


\[
\Phi_k(x) = e^{\lambda_k x} v_k.
\]


Тогда


\[
\Phi_k'(x) = \lambda_k e^{\lambda_k x} v_k,
\]


а


\[
A\Phi_k(x) = A(e^{\lambda_k x} v_k)
           = e^{\lambda_k x} A v_k
           = e^{\lambda_k x} \lambda_k v_k
           = \lambda_k e^{\lambda_k x} v_k
           = \Phi_k'(x).
\]


Следовательно, $\Phi_k'(x) = A \Phi_k(x)$, то есть $\Phi_k(x)$ — решение системы \eqref{7.3} для каждого $k=1,\ldots,n$.

\textbf{Шаг 2. Линейная независимость $\{\Phi_k(x)\}_{k=1}^n$.}

Пусть для некоторого фиксированного $x$ имеем линейную комбинацию


\[
\alpha_1 \Phi_1(x) + \alpha_2 \Phi_2(x) + \ldots + \alpha_n \Phi_n(x) = 0,
\]


то есть


\[
\alpha_1 e^{\lambda_1 x} v_1 + \alpha_2 e^{\lambda_2 x} v_2 + \ldots + \alpha_n e^{\lambda_n x} v_n = 0.
\]


Так как все $e^{\lambda_k x} \neq 0$, можно переписать как


\[
\beta_1 v_1 + \beta_2 v_2 + \ldots + \beta_n v_n = 0,
\quad \text{где } \beta_k = \alpha_k e^{\lambda_k x}.
\]


Но по предположению $v_1,\ldots,v_n$ — линейно независимая система векторов, значит


\[
\beta_1 = \beta_2 = \ldots = \beta_n = 0.
\]


Отсюда $\alpha_k = 0$ для всех $k$, следовательно, векторы–функции $\Phi_1(x),\ldots,\Phi_n(x)$ линейно независимы как вектор–функции на любом интервале.

\textbf{Шаг 3. Фундаментальная система и вронскиан.}

Мы получили $n$ решений $\Phi_k(x)$, линейно независимых. По определению 7.1 и замечанию 7.1, система $\{\Phi_k(x)\}_{k=1}^n$ является фундаментальной системой решений СЛДУ \eqref{7.3}. По теореме 7.3 вронскиан $W(x)$ этой системы не равен тождественно нулю и, более того, $W(x)\neq 0$ для всех $x$ на рассматриваемом интервале.

\textbf{Шаг 4. Общий вид решения.}

По определению общего решения (Определение 7.4) и теореме о существовании и единственности (Теорема 7.2), для однородной системы порядка $n$ общее решение есть $n$–параметрическое семейство, получаемое как линейная комбинация фундаментальной системы:


\[
Y(x) = c_1 \Phi_1(x) + \ldots + c_n \Phi_n(x),
\]


то есть


\[
Y(x) = c_1 e^{\lambda_1 x} v_1 + c_2 e^{\lambda_2 x} v_2 + \ldots + c_n e^{\lambda_n x} v_n.
\]


Любое решение однородной системы может быть представлено в таком виде (по линейности и свойству ФСР), и разные наборы $(c_1,\ldots,c_n)$ дают различные решения (из линейной независимости $\{\Phi_k\}$).

Это и есть общий вид решения однородной системы ЛДУ в случае, когда количество линейно независимых собственных векторов матрицы $A$ совпадает с порядком системы.

	\newpage
}