{
	\section{Теорема о построении ФСР однородного ЛДУ с постоянными коэффициентами, если известны корни его
	характеристичекого многочлена.}

	\subsection*{Определение 5.2}

	Многочлен


	\[
	D(t) = t^n + p_{n-1} t^{n-1} + \ldots + p_1 t + p_0
	\]


	называется \textbf{характеристическим многочленом} уравнения (5.1) или оператора \( L \).

	\[
		y^{(n)} + p_{n-1} y^{(n-1)} + \ldots + p_1 y' + p_0 y = 0 \quad \iff \quad L(y) = 0 \tag{5.1}
	\]

	\subsection*{Теорема 5.2}

	\begin{enumerate}
	\item Пусть \( \lambda \) — корень характеристического многочлена \( D(t) \) кратности \( k \). Тогда функции
	

	\[
	e^{\lambda x},\quad x e^{\lambda x},\quad \ldots,\quad x^{k-1} e^{\lambda x}
	\]


	являются решениями однородного ЛДУ (5.1).

	\item Если \( L(x^p e^{\lambda x}) = 0 \) при \( x = 0 \) для \( p = 0, 1, \ldots, k-1 \), то \( \lambda \) — корень характеристического многочлена \( D(t) \) кратности не менее, чем \( k \).

	\item Пусть \( \lambda_1, \lambda_2, \ldots, \lambda_m \) — корни характеристического многочлена \( D(t) \), а \( k_1, k_2, \ldots, k_m \) — их кратности, такие что
	

	\[
	k_1 + k_2 + \ldots + k_m = n.
	\]


	Тогда функции
	

	\[
	e^{\lambda_1 x},\quad x e^{\lambda_1 x},\quad \ldots,\quad x^{k_1 - 1} e^{\lambda_1 x},
	\]


	

	\[
	e^{\lambda_2 x},\quad x e^{\lambda_2 x},\quad \ldots,\quad x^{k_2 - 1} e^{\lambda_2 x},
	\]


	

	\[
	\ldots
	\]


	

	\[
	e^{\lambda_m x},\quad x e^{\lambda_m x},\quad \ldots,\quad x^{k_m - 1} e^{\lambda_m x}
	\]


	образуют фундаментальную систему решений (ФСР) уравнения (5.1).
	\end{enumerate}


	\subsection*{Примеры}

	\begin{enumerate}
	\item Уравнение:
	

	\[
	y''' - 2y'' + y' - 2y = 0
	\]


	Характеристический многочлен:
	

	\[
	D(t) = t^3 - 2t^2 + t - 2 = (t - 2)(t - 1)(t + 1)
	\]


	Корни: \( \lambda_1 = 2, \lambda_2 = 1, \lambda_3 = -1 \) — кратность 1.

	ФСР: \( e^{2x},\ e^x,\ e^{-x} \)

	Общее решение:
	

	\[
	y(x) = c_1 e^{2x} + c_2 e^x + c_3 e^{-x}
	\]



	\item Уравнение:
	

	\[
	y'' + 2y' + y = 0
	\]


	Характеристический многочлен:
	

	\[
	D(t) = t^2 + 2t + 1 = (t + 1)^2
	\]


	Корень: \( \lambda = -1 \), кратность 2.

	ФСР: \( e^{-x},\ x e^{-x} \)

	Общее решение:
	

	\[
	y(x) = c_1 e^{-x} + c_2 x e^{-x}
	\]



	\item Уравнение:
	

	\[
	y^{(4)} + 2y'' = 0
	\]


	Характеристический многочлен:
	

	\[
	D(t) = t^4 + 2t^2 = t^2(t^2 + 2)
	\]


	Корни: \( \lambda_{1,2} = \pm i\sqrt{2} \), кратность 1; \( \lambda = 0 \), кратность 2.

	ФСР: \( e^{i\sqrt{2}x},\ e^{-i\sqrt{2}x},\ x,\ x^2 \)

	Общее решение:
	

	\[
	y(x) = c_1 e^{i\sqrt{2}x} + c_2 e^{-i\sqrt{2}x} + c_3 x + c_4 x^2
	\]


	\end{enumerate}


	\newpage
}