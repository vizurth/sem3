{
	\section{Линейные дифференциальные уравнения n-го порядка (ЛДУ). Теорема о существовании и единственности
	решения задачи Коши.}

	\subsection*{Определение 3.2}

	ДУ вида


	\[
	a_n(x)y^{(n)} + a_{n-1}(x)y^{(n-1)} + \ldots + a_1(x)y' + a_0(x)y = g(x)
	\]


	называется ЛДУ \( n \)-го порядка (функции \( a_n(x), a_{n-1}(x), \ldots, a_1(x), a_0(x) \), которые называются коэффициентами уравнения, функция \( g(x) \), которая называется правой частью уравнения, непрерывны на промежутке \( (a, b) \), и \( a_n(x) \neq 0 \) на \( (a, b) \)).

	Разделим обе части уравнения на \( a_n(x) \), получим


	\[
	y^{(n)} + p_{n-1}(x)y^{(n-1)} + \ldots + p_1(x)y' + p_0(x)y = q(x) \quad \text{(3.1)}.
	\]

	\subsection*{Теорема 3.2}

	Пусть \( p_{n-1}(x), \ldots, p_1(x), p_0(x) \) и \( q(x) \) непрерывны на \( (a, b) \). Тогда для любого набора значений \( (x_0, y_0, \ldots, y_{n-1}) \), где \( x_0 \in (a, b) \), \( (y_0, \ldots, y_{n-1}) \in \mathbb{R}^n \), существует единственное решение \( y(x) \) ЛДУ (3.1), удовлетворяющее начальным условиям:
	
	
	\[
	y(x_0) = y_0, \quad y'(x_0) = y_1, \quad \ldots, \quad y^{(n-1)}(x_0) = y_{n-1}.
	\]	

	\subsection*{Замечание}

	Можно показать, что для ЛДУ каждое решение определено на всём промежутке \( (a, b) \) (без доказательства).



	\newpage
}