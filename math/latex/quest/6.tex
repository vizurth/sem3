{
	\section{Линейно зависимые и независимые системы функций. Определение фундаментальной системы решений (ФСР)
	однородного ЛДУ. Теорема о свойствах ФСР.}

	\subsection*{Однородные ЛДУ}

	Общее выражение однородного линейного дифференциального уравнения порядка \( n \):

	\begin{equation}
	y^{(n)} + p_{n-1}(x)y^{(n-1)} + \ldots + p_1(x)y' + p_0(x)y = 0 \tag{4.1}
	\end{equation}

	\subsection*{Замечание}

	Множество \( p \) раз дифференцируемых на \( (a, b) \) функций образует бесконечномерное линейное пространство.

	Рассмотрим линейный дифференциальный оператор:


	\[
	L(g) = g^{(p)} + \rho_{p-1}(x)g^{(p-1)} + \ldots + \rho_1(x)g' + \rho_0(x)g
	\]



	Уравнение (4.1):


	\[
	L(g) = 0
	\]



	То есть, решения уравнения (4.1) — это функции, на которые действует дифференциальный оператор \( L \) порядка \( p \). Покажем, что \( \dim \ker L = p \).

	Напомним: оператор \( L(g) \) линейный, если


	\[
	L(g_1(x) + g_2(x)) = L(g_1) + L(g_2), \quad L(\lambda g) = \lambda L(g)
	\]

	\subsection*{Определение 4.1}

	\begin{enumerate}
	\item Функции \( \varphi_1(x), \varphi_2(x), \ldots, \varphi_k(x) \), определённые на \( (a, b) \), называются \textbf{линейно независимыми} на \( (a, b) \), если равенство
	

	\[
	c_1 \varphi_1(x) + c_2 \varphi_2(x) + \ldots + c_k \varphi_k(x) = 0 \quad \text{на } (a, b)
	\]


	возможно только в случае, когда \( c_1 = c_2 = \ldots = c_k = 0 \).

	\item Функции \( \varphi_1(x), \varphi_2(x), \ldots, \varphi_k(x) \), определённые на \( (a, b) \), называются \textbf{линейно зависимыми} на \( (a, b) \), если существуют числа \( c_1, c_2, \ldots, c_k \), не все из которых равны нулю, такие, что выполняется равенство
	

	\[
	c_1 \varphi_1(x) + c_2 \varphi_2(x) + \ldots + c_k \varphi_k(x) = 0 \quad \text{на } (a, b).
	\]


	\end{enumerate}

	\subsection*{Следствия.}


	\begin{enumerate}
		\item Система функций \( \left\{ \varphi_i(x) \right\}_{i=1}^{k} \) линейно зависима (ЛЗ) \(\iff\) одна из них является линейной комбинацией (ЛК) остальных.
		
		\item Система функций \( \left\{ \varphi_i(x) \right\}_{i=1}^{k} \) содержит \textbf{\(0\)}, то она линейно зависима.
	\end{enumerate}

	\subsection*{Определение 4.2.}

	Функции $\varphi_{1}(x)$, $\varphi_{2}(x)$, \ldots, $\varphi_{n}(x)$,
	определённые на $(a,b)$, называются \emph{фундаментальной системой решений} (ФСР)
	уравнения (4.1), если

	1) функции $\varphi_{1}(x)$, $\varphi_{2}(x)$, \ldots, $\varphi_{n}(x)$ являются решениями уравнения;

	2) их количество совпадает с порядком уравнения;

	3) $\{\varphi_{k}(x)\}_{k=1}^{n}$ — линейно независимая (ЛНЗ) система функций.



	\subsection*{Определение 4.3}

	Пусть функции \( \varphi_1(x), \varphi_2(x), \ldots, \varphi_n(x) \) определены на интервале \( (a, b) \).

	Определим функцию


	\[
	W(x) = 
	\begin{vmatrix}
	\varphi_1(x) & \varphi_2(x) & \ldots & \varphi_n(x) \\
	\varphi_1'(x) & \varphi_2'(x) & \ldots & \varphi_n'(x) \\
	\vdots & \vdots & \ddots & \vdots \\
	\varphi_1^{(n-1)}(x) & \varphi_2^{(n-1)}(x) & \ldots & \varphi_n^{(n-1)}(x)
	\end{vmatrix}
	\]



	Функция \( W(x) \), определённая на \( (a, b) \), называется \textbf{вронскианом} системы функций \( \left\{ \varphi_k(x) \right\}_{k=1}^{n} \).


	\subsection*{Теорема 4.1}

	Пусть все коэффициенты \( \rho_{n-1}(x), \ldots, \rho_1(x), \rho_0(x) \) уравнения


	\[
	y^{(n)} + \rho_{n-1}(x)y^{(n-1)} + \ldots + \rho_1(x)y' + \rho_0(x)y = 0 \tag{4.1}
	\]


	непрерывны на интервале \( (a, b) \).

	Пусть функции \( \varphi_1(x), \varphi_2(x), \ldots, \varphi_n(x) \) являются решениями этого уравнения.

	Следующие три утверждения равносильны:

	\begin{enumerate}
		\item \( \left\{ \Phi(x) \right\}_{x=1}^{n} \) — фундаментальная система решений (ФСР) дифференциального уравнения.
	  
		\item Вронскиан системы функций \( W(x) \) не равен тождественно нулю на интервале \( (a, b) \), то есть
		
	  
	  \[
		\exists \, x \in (a, b) : W(x) \neq 0.
		\]
	  
	  
	  
		\item Вронскиан системы функций \( W(x) \) не равен нулю на всём интервале \( (a, b) \), то есть
		
	  
	  \[
		W(x) \neq 0 \quad \forall x \in (a, b).
		\]
	  
	  
	  \end{enumerate}


	\subsection*{Следствие}

	Пусть функции \( \varphi_1(x), \varphi_2(x), \ldots, \varphi_n(x) \) являются решениями дифференциального уравнения. Тогда:

	\begin{enumerate}
	\item либо вронскиан системы функций \( W(x) = 0 \quad \forall x \in (a, b) \), что равносильно тому, что \( \left\{ \varphi_k(x) \right\}_{k=1}^{n} \) — линейно зависимая система функций;

	\item либо \( W(x) \neq 0 \quad \forall x \in (a, b) \), что равносильно тому, что \( \left\{ \varphi_k(x) \right\}_{k=1}^{n} \) — линейно независимая система функций.
	\end{enumerate}

	\newpage
}