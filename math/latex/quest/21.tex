{
	\setcounter{section}{20}
	\section{Понятие числового ряда. Асимптотическая формула для частичной суммы гармонического ряда.}

	\section*{Определение 1.1}

Пусть $\{a_n\}_{n=1}^{\infty}$ — числовая последовательность.

Символ


\[
\sum_{n=1}^{\infty} a_n = a_1 + a_2 + \ldots + a_n + \ldots
\]


называется \textbf{числовым рядом} или \textbf{бесконечной суммой}, а $a_n$ — \textbf{общим членом ряда}.

\medskip

\textbf{Частичной суммой} ряда называется сумма


\[
S_n = \sum_{k=1}^{n} a_k.
\]



Если существует конечный предел последовательности частичных сумм $\{S_n\}_{n=1}^{\infty}$, равный $S$, то символу $\sum_{n=1}^{\infty} a_n$ приписывается значение $S$, которое называется \textbf{суммой ряда}, а сам ряд называется \textbf{сходящимся} (к $S$).

Если предел последовательности частичных сумм $\{S_n\}_{n=1}^{\infty}$ бесконечен или не существует, ряд $\sum_{n=1}^{\infty} a_n$ называется \textbf{расходящимся}.

\section*{Лемма 1.1. Асимптотическая формула для частичных сумм гармонического ряда}

Рассмотрим гармонический ряд:


\[
\sum_{n=1}^{\infty} \frac{1}{n}.
\]



Его частичная сумма имеет асимптотическое представление:


\[
S_n = c + \ln n + \alpha_n,
\]


где


\[
c = 0{,}5772\ldots
\]


— \textbf{постоянная Эйлера} (или \emph{константа Эйлера–Маскерони}), а


\[
\alpha_n \to 0 \quad \text{при} \quad n \to \infty.
\]

\section*{Следствие}




Гармонический  \(\sum_{n=1}^{\infty} \frac{1}{n}\) ряд \textbf{расходится}, так как


\[
S_n = c + \ln n + \alpha_n \xrightarrow[n \to \infty]{} \infty.
\]



\section*{Пример. Сумма геометрической прогрессии}

Рассмотрим ряд


\[
\sum_{n=0}^{\infty} q^n.
\]



Частичная сумма:


\[
S_n = 1 + q + q^2 + \ldots + q^{n-1} =
\begin{cases}
\displaystyle \frac{1 - q^n}{1 - q}, & q \neq 1, \\
n, & q = 1.
\end{cases}
\]



Так как


\[
q^n \xrightarrow[n \to \infty]{} 0, \quad \text{если } |q| < 1; \qquad
q^n \xrightarrow[n \to \infty]{} \infty, \quad \text{если } |q| > 1;
\]




\[
q^n \text{ не имеет предела, если } q = -1,
\]


то ряд


\[
\sum_{n=0}^{\infty} q^n
\]


\textbf{сходится} к \( \displaystyle \frac{1}{1 - q} \) при \( |q| < 1 \), и \textbf{расходится} при \( |q| \geq 1 \).


	\newpage
}