{
	\section{Понятие числового ряда. Асимптотическая формула для частичной суммы гармонического ряда.}

	\section*{Определение 1.1}

Пусть $\{a_n\}_{n=1}^{\infty}$ — числовая последовательность.

Символ


\[
\sum_{n=1}^{\infty} a_n = a_1 + a_2 + \ldots + a_n + \ldots
\]


называется \textbf{числовым рядом} или \textbf{бесконечной суммой}, а $a_n$ — \textbf{общим членом ряда}.

\medskip

\textbf{Частичной суммой} ряда называется сумма


\[
S_n = \sum_{k=1}^{n} a_k.
\]



Если существует конечный предел последовательности частичных сумм $\{S_n\}_{n=1}^{\infty}$, равный $S$, то символу $\sum_{n=1}^{\infty} a_n$ приписывается значение $S$, которое называется \textbf{суммой ряда}, а сам ряд называется \textbf{сходящимся} (к $S$).

Если предел последовательности частичных сумм $\{S_n\}_{n=1}^{\infty}$ бесконечен или не существует, ряд $\sum_{n=1}^{\infty} a_n$ называется \textbf{расходящимся}.

	\newpage
}