{
	\section{Общее решение однородной системы ЛДУ в случае, когда количество линейно-
	независимых собственных векторов матрицы системы меньше порядка системы.}

	Рассмотрим однородную систему линейных дифференциальных уравнений


\[
Y'(x) = A Y(x),
\]


где $A$ — постоянная матрица порядка $n$.

Если матрица $A$ имеет \textbf{менее $n$ линейно независимых собственных векторов}, то она не диагонализируема и приводится к жордановой форме:


\[
A = S J S^{-1},
\]


где $J$ — жорданова матрица, состоящая из жордановых блоков


\[
J_\alpha =
\begin{pmatrix}
\lambda_\alpha & 1           & 0           & \cdots & 0 \\
0              & \lambda_\alpha & 1       & \cdots & 0 \\
\vdots         & \vdots      & \ddots      & \ddots & \vdots \\
0              & 0           & \cdots      & \lambda_\alpha & 1 \\
0              & 0           & \cdots      & 0            & \lambda_\alpha
\end{pmatrix},
\]


каждый из которых соответствует собственному значению $\lambda_\alpha$ и имеет размер $k_\alpha \times k_\alpha$.

Для жорданова блока размера $k_\alpha$ фундаментальная система решений имеет вид


\[
e^{\lambda_\alpha x} v_{\alpha,0},\quad
e^{\lambda_\alpha x}(x v_{\alpha,0} + v_{\alpha,1}),\quad
\ldots,\quad
e^{\lambda_\alpha x}\left(\frac{x^{k_\alpha-1}}{(k_\alpha-1)!} v_{\alpha,0}
+ \cdots + v_{\alpha,k_\alpha-1}\right),
\]


где $v_{\alpha,0}, v_{\alpha,1}, \ldots, v_{\alpha,k_\alpha-1}$ — цепочка обобщённых собственных векторов, соответствующая блоку $J_\alpha$.

\medskip

\textbf{Общее решение} системы $Y'(x)=A Y(x)$ имеет вид


\[
Y(x) = \sum_{\alpha} \sum_{j=0}^{k_\alpha-1}
c_{\alpha j}\, e^{\lambda_\alpha x} P_{\alpha j}(x),
\]


где $P_{\alpha j}(x)$ — вектор–многочлены по $x$ степени не выше $k_\alpha-1$, а $c_{\alpha j}$ — произвольные постоянные.

Всего таких независимых решений $n$, и они образуют фундаментальную систему решений.

	\newpage
}