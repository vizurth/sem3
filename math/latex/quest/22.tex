{
	\section{Теоремы о сходящихся рядах (возможность заключать элементы в скобки; сходимость ряда с элементами -
	линейными комбинациями элементов сходящихся рядов). (с доказательством)}

\subsection*{Теорема 1.1 возможность заключать элементы в скобки}

Члены сходящегося ряда можно заключать в скобки, то есть:

Пусть ряд


\[
\sum_{n=1}^{\infty} a_n
\]


сходится к $S$ (то есть его сумма равна $S$). Тогда ряд


\[
\sum_{k=1}^{\infty} b_k, \quad \text{где } b_k = a_{n_{k-1}+1} + a_{n_{k-1}+2} + \ldots + a_{n_k}
\]


(то есть члены исходного ряда сгруппированы в блоки $(a_1 + \ldots + a_{n_1}), (a_{n_1+1} + \ldots + a_{n_2}), \ldots$)

\textbf{тоже сходится} и имеет ту же сумму $S$:


\[
\sum_{k=1}^{\infty} b_k = S.
\]

\subsection*{Доказательство}

Частичные суммы рядов связаны равенством


\[
S_k^{(s)} = S_{n_k}^{(a)}.
\]



Следовательно, последовательность частичных сумм ряда


\[
\sum_{n=1}^{\infty} b_n
\]


является подпоследовательностью последовательности частичных сумм ряда


\[
\sum_{n=1}^{\infty} a_n
\]


и имеет тот же предел $S$.

\subsection*{Замечание}

Обратное неверно. Например, рассмотрим ряд


\[
\sum_{n=1}^{\infty} 0 = 0.
\]



Этот ряд можно рассматривать как


\[
(1 - 1) + (1 - 1) + (1 - 1) + \ldots
\]



Если раскроем скобки, получим ряд


\[
\sum_{n=1}^{\infty} (-1)^{n-1},
\]


который расходится.

\subsection*{Теорема 1.2 сходимость ряда с элементами - линейными комбинациями элементов сходящихся рядов}

Пусть ряд


\[
\sum_{n=1}^{\infty} a_n
\]


сходится к $S^{(a)}$, а ряд


\[
\sum_{n=1}^{\infty} b_n
\]


сходится к $S^{(b)}$.

Тогда ряд


\[
\sum_{n=1}^{\infty} (\alpha a_n + \beta b_n)
\]


сходится, и его сумма равна


\[
\alpha S^{(a)} + \beta S^{(b)}.
\]



\subsection*{Доказательство}

Частичная сумма ряда


\[
S_n^{(3)} = \sum_{k=1}^{n} (\alpha a_k + \beta b_k)
= \alpha \sum_{k=1}^{n} a_k + \beta \sum_{k=1}^{n} b_k
= \alpha S_n^{(1)} + \beta S_n^{(2)}.
\]



Так как $S_n^{(1)} \to S^{(a)}$ и $S_n^{(2)} \to S^{(b)}$ при $n \to \infty$, то


\[
S_n^{(3)} \to \alpha S^{(a)} + \beta S^{(b)}.
\]

\subsection*{Следствия}

\textbf{1)} Постоянный множитель можно выносить за знак суммы ряда, то есть:

Если


\[
\sum_{n=1}^{\infty} a_n
\]


сходится к $S^{(a)}$, то


\[
\sum_{n=1}^{\infty} \alpha a_n
\]


сходится к $\alpha S^{(a)}$.

(Это частный случай Теоремы 1.2 при $\beta = 0$.)

\medskip

\textbf{2)} Если ряд (1) сходится, а ряд (2) расходится, то ряд (3)


\[
\sum_{n=1}^{\infty} (\alpha a_n + \beta b_n)
\]


\textbf{расходится} при $\beta \ne 0$, так как последовательность


\[
S_n^{(2)} = \frac{1}{\beta} S_n^{(3)} - \frac{\alpha}{\beta} S_n^{(1)}
\]


была бы сходящей, если бы сходилась последовательность


\[
S_n^{(3)}.
\]


	\newpage
}