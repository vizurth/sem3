{
	\section{Уравнение с разделяющимися переменными. Линейное уравнение 1-го порядка. Уравнение в полных
	дифференциалах.}

	\subsection*{Некоторые классы нормальных ДУ 1-го порядка.}

	
	Рассмотрим нормальное ДУ первого порядка:


	\[
	y' = f(x, y) \tag{6}
	\]



	Другая записи:


	\[
	M(x, y)\, dx + N(x, y)\, dy = 0
	\]



	Этот вид объединяет два уравнения:


	\[
	y' = -\dfrac{M(x, y)}{N(x, y)} \quad \text{и} \quad x' = -\dfrac{N(x, y)}{M(x, y)}
	\]



	\vspace{0.5em}
	\noindent\textbf{Напоминание:} по определению производной


	\[
	\frac{dy}{dx} = y'(x)\, dx \quad \Rightarrow \quad y' = \frac{dy}{dx}
	\]

	\subsection*{I. Уравнение с разделяющимися переменными}

	Пусть дана функция $g(x)f(y)$, где $g(x)f(y)$ непрерывна на $(a, b) \times (c, d) = \Omega$.
	
	\begin{enumerate}
		\item Рассмотрим область $\Omega^* \subset \Omega$, в которой $g_2(y) \neq 0$.



		\[
		g_1(x)dx - \left( \frac{1}{g_2(y)} \right)dy = 0 \Leftrightarrow d(G(x) - F(y)) = 0 \Leftrightarrow G(x) - F(y) = \text{const} \quad \text{(всюду в области } \Omega^* \subset \Omega, \text{общий интеграл)}
		\]
		
		(Здесь $G(y)$ — некоторая фиксированная первообразная функции $g_1(y)$;
		Здесь $F(y)$ — некоторая фиксированная первообразная функции $1 / g_2(y)$.)
		
		
		Если $g_2(y)$ непрерывно дифференцируема, уравнение удовлетворяет теореме о существовании и единственности решения задачи Коши. Следовательно, через каждую точку $\Omega^*$ проходит ровно одна интегральная кривая. Также выполнены требования теоремы 2.1. Следовательно, равенство $G(x) - F(y) = \text{const}$ содержит все решения   ДУ   в области $\Omega^*$.

		\item Если $\exists\, c : g_2(c^*) = 0$, то $y(x) = c^*$ — решение ДУ.

		\item Другая запись ДУ с разделяющимися переменными:

		\[
		M_1(x)M_2(y) \, dx + N_1(x)N_2(y) \, dy = 0 \quad \Leftrightarrow \quad
		\begin{cases}
		y' = -\dfrac{M_1(x)M_2(y)}{N_1(x)N_2(y)} \\
		x' = -\dfrac{N_1(x)N_2(y)}{M_1(x)M_2(y)}
		\end{cases}
		\]
		
		
		
		В этом случае, если \( \exists \, c^* : N_1(c^*) = 0 \), то \( x(y) = c^* \) также будет решением ДУ.

		\textbf{Пример.}



	\[
	y' = \frac{xy}{x + 1}
	\]



	Ответ: $y = \frac{ce^x}{x + 1}$ — общий интеграл (содержит все решения).

	\textbf{Замечания к примеру.}

	a. Если бы пример был записан в виде $xydx - (x + 1) \, dy = 0$, то добавилось бы решение $x(y) = -1$.

	b. Найдём решение, удовлетворяющее начальным условиям $y(0) = -2$.

	Отметим, что в каждой из полуплоскостей $x > -1$, $x < -1$ выполнены требования теоремы о существовании и единственности решения задачи Коши. Следовательно, через каждую точку полуплоскости проходит ровно одна интегральная кривая.

	Обозначим начальные условия: $-2 = c \to c = -2 \to y = \frac{-2e^x}{x + 1}$ — особое решение, определенное на $(-1; + \infty)$.

		
	\end{enumerate}

	\newpage

	\subsection*{II. Линейное ДУ 1-го порядка}



	\[
	y' + p(x)y = g(x), \quad (7), \quad \text{где } p(x) \text{ и } g(x) \text{ непрерывны на } (a, b).
	\]





	\[
	y' = -p(x)y + g(x).
	\]





	\[
	f(x, y) = -p(x)y + g(x), \quad \frac{\partial f}{\partial y} = -p(x)
	\]



	непрерывны в области \( G = (a, b) \times \mathbb{R} \). Следовательно, выполнены требования теоремы о существовании и единственности решения задачи Коши. Следовательно, через каждую точку области \( G \) проходит ровно одна интегральная кривая.

	Следовательно, \( g(x) = 0 \) и \( f(x, y) = -p(x)y \).

	\begin{enumerate}
		\item Решим однородное уравнение \( y' + p(x)y = 0 \).

		Это уравнение с разделяющимися переменными.



		\[
		\frac{dy}{y} = -p(x)dx ; \quad \ln |y| = \Phi(x) + c_1\] 
		(где  $\Phi(x)$ - некоторая фиксированная первообразная функции $-p(x)$ на $(a, b), c_1$ - произвольная постоянная);





		\[
		|y| = e^{\Phi(x)} \cdot c_2 \quad \text{(где } c_2 = e^{c_1} > 0); \quad y = c_3 \cdot e^{\Phi(x)} \quad \text{(где } c_3 \neq 0);
		\]



		потеряли решение \( y(x) = 0 \). Следовательно, общий интеграл \( y = c \cdot e^{\Phi(x)} \) ($c$ произвольной постоянной); также называется общим решением однородного линейного уравнения; содержит все решения (так как разрешим относительно параметра \( c \)).

		\item Будем искать решение неоднородного уравнения в виде \( y = e^{F(x)} c(x) \), где \( c(x) \) — неизвестная функция.

		Подставим в уравнение (7):

		Так как \( y' = e^{\Phi(x)} c(x)\Phi'(x) + e^{\Phi(x)} c'(x) = -p(x) e^{\Phi(x)}c(x) + e^{\Phi(x)}c'(x)\), то



		\[
		-p(x) e^{\Phi(x)} c(x) + e^{F(x)} c'(x) + p(x) e^{\Phi(x)} c(x) = g(x); \quad c'(x) = e^{-\Phi(x)} g(x);
		\]





		\[
		c(x) = F(x) + c^*
		\]

		(где $F(x)$ — некоторая фиксированная первообразная функция $e^{-\Phi(x)}g(x)$ на $(a, b)$, $c^*$ - произвольная постоянная)

		Следовательно, \( y(x) = e^{\Phi(x)} F(x) + e^{\Phi(x)} c^* \) — общий интеграл, также называется общим решением неоднородного линейного уравнения; содержит все решения (так как разрешим относительно параметра \( c^* \)).

	Отметим, что первое слагаемое \( e^{\Phi(x)} F(x) \) — это частное решение неоднородного уравнения, второе слагаемое \( e^{\Phi(x)} c^* \) — это общее решение соответствующего однородного уравнения.

	\end{enumerate}

	\subsection*{III. Уравнение в полных дифференциалах}



	\[
	P(x, y)\,dx + Q(x, y)\,dy = 0; \quad P(x, y),\; Q(x, y) \text{ непрерывны в области } D \subset \mathbb{R}^2.
	\]



	Это уравнение называется уравнением в полных дифференциалах, если существует непрерывно дифференцируемая в \( D \) функция \( u(x, y) \): \( du = P\,dx + Q\,dy \). в \( D \)

	В этом случае равенство \( u(x,y) = c \) является общим интегралом уравнения, так как



	\[
	\frac{\partial u}{\partial x}\,dx + \frac{\partial u}{\partial y}\,dy = 0
	\]




	\[
	P\,dx + Q\,dy = 0.
	\]



	Общий интеграл \( u(x, y) = c \) содержит все решения (так как разрешен относительно параметра \( c \)).

	\subsection*{Определение 2.3}

	\begin{enumerate}
	\item Множество \( D \subset \mathbb{R}^2 \) называется связным, если любые две точки из него можно соединить непрерывной кривой, целиком лежащей в \( D \).
	
	\item Связное множество \( D \subset \mathbb{R}^2 \) называется односвязным, если любую замкнутую непрерывную кривую в \( D \), как бы она ни была взята, можно стянуть в точку непрерывным образом, не выходя из \( D \).
	
	\item Открытое связное множество называется областью.
	\end{enumerate}

	\subsection*{Лемма 2.1}

	Пусть \( D \subset \mathbb{R}^2 \) — односвязная область, и в \( D \) существуют и непрерывны \( \frac{\partial Q}{\partial x} \) и \( \frac{\partial P}{\partial y} \). Тогда, для того, чтобы уравнение \( P(x, y)\,dx + Q(x, y)\,dy = 0 \) было уравнением в полных дифференциалах, необходимо и достаточно, чтобы


	\[
	\frac{\partial Q}{\partial x} = \frac{\partial P}{\partial y} \quad \text{в} \quad D.
	\]

}