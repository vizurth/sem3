{
	\section{Функция $e^z$ и далее переписать из списка}

\subsection*{Определение 8.1.}

Рассмотрим ряд
\[
\sum_{n=0}^{\infty} \frac{z^n}{n!}.
\]

Ряд
\[
\sum_{n=0}^{\infty} \frac{|z|^n}{n!}
\]
сходится $\ \forall z \in \mathbb{C}$, так как ряд
\[
\sum_{n=0}^{\infty} \frac{x^n}{n!}
\]
сходится абсолютно

$\forall x \in \mathbb{R}$, и
\[
\forall z \in \mathbb{R} \quad
\sum_{n=0}^{\infty} \frac{z^n}{n!} = e^{z}.
\]

Назовем сумму этого ряда $S(z)$ значением функции $e^{z}$
$\ \forall z \in \mathbb{C}$

( мы продолжили функцию $e^{x}$ с действительной оси
на всю комплексную плоскость ).

\subsection*{Теорема 8.1.}

Осталось справедливым свойство экспоненты
\[
e^{z_1} e^{z_2} = e^{z_1 + z_2}.
\]

\medskip

\subsection*{Доказательство.}

\medskip

\textit{Напоминание:}

\medskip

\subsection*{Теорема 3.4.}

Пусть ряды
\[
\sum_{n=0}^{\infty} a_n
\quad \text{и} \quad
\sum_{n=0}^{\infty} b_n
\]
сходятся абсолютно, и их суммы есть $A$ и $B$. Тогда
ряд
\[
\sum_{n=0}^{\infty} c_n,
\]
где
\[
c_n = a_0 b_n + a_1 b_{n-1} + \dots + a_n b_0,
\]
также сходится абсолютно, и его сумма равна $AB$.

\medskip

Имеем
\[
e^{z_1} = \sum_{n=0}^{\infty} \frac{z_1^{\,n}}{n!},
\quad
e^{z_2} = \sum_{n=0}^{\infty} \frac{z_2^{\,n}}{n!}.
\]
Тогда
\[
e^{z_1} e^{z_2} = \sum_{n=0}^{\infty} c_n,
\]
где
\[
c_n =
\frac{z_2^{\,n}}{n!}
+ \frac{z_1 z_2^{\,n-1}}{1!(n-1)!}
+ \frac{z_1^{\,2} z_2^{\,n-2}}{2!(n-2)!}
+ \dots
+ \frac{z_1^{\,p} z_2^{\,n-p}}{p!(n-p)!}
+ \dots
+ \frac{z_1^{\,n}}{n!}
\]

\[
= \frac{1}{n!}
\left(
z_2^{\,n}
+ \frac{n!}{1!(n-1)!} z_1 z_2^{\,n-1}
+ \dots
+ \frac{n!}{p!(n-p)!} z_1^{\,p} z_2^{\,n-p}
+ \dots
+ z_1^{\,n}
\right)
= \frac{(z_1 + z_2)^n}{n!}.
\]

Следовательно,
\[
e^{z_1} e^{z_2}
= \sum_{n=0}^{\infty} \frac{(z_1 + z_2)^n}{n!}
= e^{z_1 + z_2}.
\]




	\newpage
}