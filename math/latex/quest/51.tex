{
	\section{Тригонометрический ряд Фурье. Формулы для его коэффициентов. Свойства ряда Фурье, вытекающие из
	полноты тригонометрической системы функций.}
\section*{Определение 10.1.}

Рассмотрим пространство $L_2([-\pi,\pi])$.

\[
e =
\left\{
\frac{1}{\sqrt{2\pi}},
\frac{\cos x}{\sqrt{\pi}},
\frac{\sin x}{\sqrt{\pi}},
\ldots,
\frac{\cos nx}{\sqrt{\pi}},
\frac{\sin nx}{\sqrt{\pi}},
\ldots
\right\}
\]
— ортонормированная система функций в
$L_2([-\pi,\pi])$.

\medskip

Сопоставим каждой функции
$f \in L_2([-\pi,\pi])$
ряд Фурье этой функции по системе $e$,
который называется тригонометрическим
и записывается следующим образом
\[
f \sim
\frac{a_0}{2}
+
\sum_{n=1}^{\infty}
\left(
a_n \cos nx + b_n \sin nx
\right).
\]

\medskip

\section*{Лемма 10.1.}

\medskip

Коэффициенты тригонометрического ряда Фурье имеют вид

\[
a_0 =
\frac{1}{\pi}
\int_{-\pi}^{\pi} f(x)\,dx,
\qquad
a_n =
\frac{1}{\pi}
\int_{-\pi}^{\pi} f(x)\cos nx\,dx,
\]

\[
b_n =
\frac{1}{\pi}
\int_{-\pi}^{\pi} f(x)\sin nx\,dx,
\qquad
(n \in \mathbb{N}).
\]

\section*{Теорема 10.1.}

Система функций
\[
e =
\left\{
\frac{1}{\sqrt{2\pi}},
\frac{\cos x}{\sqrt{\pi}},
\frac{\sin x}{\sqrt{\pi}},
\ldots,
\frac{\cos nx}{\sqrt{\pi}},
\frac{\sin nx}{\sqrt{\pi}},
\ldots
\right\}
\]
является полной в
$L_2([-\pi,\pi])$, если рассматривать интеграл как интеграл Лебега.

\[
(\text{без доказательства})
\]

\medskip

\section*{Замечание.}

\medskip

Множество функций, интегрируемых по Лебегу, шире, чем множество функций,
интегрируемых по Риману.

\medskip

Например, функция Дирихле
\[
f(x) =
\begin{cases}
1, & x \in \mathbb{Q}, \\
0, & x \notin \mathbb{Q},
\end{cases}
\]
не интегрируема по Риману, но существует интеграл Лебега от этой функции
по любому отрезку и равен нулю.

\medskip

Все функции, интегрируемые по Риману, интегрируемы по Лебегу,
и значения этих интегралов совпадают.


\section*{Следствия к теореме 10.1.}

\medskip

Выполняются все утверждения предыдущего параграфа:

\medskip

1). $\forall\, f \in L_2([-\pi,\pi])$ тригонометрический ряд Фурье этой
функции сходится к $f$ по норме пространства
$L_2([-\pi,\pi])$, то есть
\[
\|f - S_n\|
=
\sqrt{
\int_{-\pi}^{\pi}
\bigl( f - S_n \bigr)^2\,dx
}
\xrightarrow[n\to\infty]{} 0,
\]
где $S_n$ — частичная сумма ряда Фурье.

Такая сходимость называется сходимостью в среднем, или
сходимостью в смысле среднего квадратичного.

\medskip

Сходимость в среднем не имеет ничего общего с поточечной
сходимостью. Существует пример непрерывной функции, ряд Фурье
которой расходится в бесконечном количестве точек из
$[-\pi,\pi]$.

\medskip

Геометрический смысл: площадь заштрихованной области стремится
к нулю при
\[
n \to \infty .
\]

\medskip

2). $\forall\, f \in L_2([-\pi,\pi])$ выполняется равенство Парсеваля
\[
\frac{1}{\pi}
\int_{-\pi}^{\pi} f^2(x)\,dx
=
\frac{(a_0)^2}{2}
+
\sum_{n=1}^{\infty}
\bigl( a_n^2 + b_n^2 \bigr).
\]

\[
(\text{Равенство Парсеваля } \|f\|^2 = \sum_{n=1}^{\infty} (f,e_n)^2)
\Longleftrightarrow
\]

\[
\Longleftrightarrow
\int_{-\pi}^{\pi} f^2(x)\,dx
=
\left( \frac{a_0 \sqrt{2\pi}}{2} \right)^2
+
\sum_{n=1}^{\infty}
\left(
(\sqrt{\pi} a_n)^2 + (\sqrt{\pi} b_n)^2
\right)
\Longleftrightarrow
\]

\[
\Longleftrightarrow
\frac{1}{\pi}
\int_{-\pi}^{\pi} f^2(x)\,dx
=
\frac{(a_0)^2}{2}
+
\sum_{n=1}^{\infty}
\bigl( a_n^2 + b_n^2 \bigr).
\]

\medskip

3).
\[
a_n \xrightarrow[n\to\infty]{} 0,
\qquad
b_n \xrightarrow[n\to\infty]{} 0.
\]

\medskip

4). Если у двух функций из $L_2([-\pi,\pi])$ все коэффициенты
тригонометрического ряда Фурье совпадают, то эти функции равны
в смысле пространства $L_2([-\pi,\pi])$, то есть принадлежат
одному классу.




	\newpage
}