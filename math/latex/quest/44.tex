{
	\section{Степенной ряд с комплексными членами, его круг сходимости.}

	\section*{Определение 7.3.}

Ряд вида
\[
\sum_{n=0}^{\infty} a_n (z - z_0)^n,
\]
где $z_0, a_0, a_1, \ldots, a_n, \ldots \in \mathbb{C}$, называется степенным
рядом с комплексными членами. Числа
$a_0, a_1, \ldots, a_n, \ldots$ называются коэффициентами
степенного ряда.

\medskip

\section*{Теорема 7.3.}

Рассмотрим ряд
\[
\sum_{n=0}^{\infty} a_n z^n.
\]

$\exists \; R \in [0, \infty] :$

\medskip

1). $\forall z \in \mathbb{C} : \ |z| < R$ \ ряд сходится абсолютно.

\medskip

2). $\forall z \in \mathbb{C} : \ |z| > R$ \ ряд расходится.

\medskip

\section*{Доказательство.}

Рассмотрим ряд
\[
\sum_{n=0}^{\infty} |a_n| \, |z|^n.
\]
Это степенной ряд с вещественными членами.

По теореме 6.1 \quad $\exists \; R \in [0, \infty] :$

\medskip

1). $\forall z : \ |z| < R$ \ ряд сходится абсолютно.

\medskip

2). $\forall z : \ |z| > R$ \ ряд расходится (так как по следствию
к теореме 6.1 $|a_n||z|^n$ не стремится к нулю, и по замечанию 1)
к определению 7.1 $a_n z^n$ не стремится к нулю).

\medskip

\section*{Определение 7.4.}

Множество
\[
D_R = \{ z \in \mathbb{C} \mid |z| < R \}
\]
называется кругом сходимости степенного ряда,

$R$ --- радиус сходимости.

	\newpage
}