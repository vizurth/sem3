{
	\section{Определение общего решения ЛДУ n-го порядка. Теорема о связи ФСР и общего решения однородного ЛДУ. (Доказательство)}

	\subsection*{Определение 4.4}

	\( n \)-параметрическая функция \( y(x) = \Psi\langle x, c_1, c_2, \ldots, c_n \rangle \) называется \textbf{общим решением} ЛДУ \( n \)-го порядка, если выполняются следующие условия:

	\begin{enumerate}
	\item для любого набора чисел \( c_1, c_2, \ldots, c_n \) функция \( y(x) \) является решением уравнения;

	\item для любого решения \( \tilde{y}(x) \) существуют такие числа \( \tilde{c}_1, \tilde{c}_2, \ldots, \tilde{c}_n \), что
	

	\[
	\tilde{y}(x) = \Psi\langle x, \tilde{c}_1, \tilde{c}_2, \ldots, \tilde{c}_n \rangle.
	\]


	\end{enumerate}



	\subsection*{Теорема 4.3}

	Пусть все коэффициенты \( p_{n-1}(x), p_{n-2}(x), \ldots, p_1(x), p_0(x) \) уравнения


	\[
	y^{(n)} + p_{n-1}(x)y^{(n-1)} + \ldots + p_1(x)y' + p_0(x)y = 0 \tag{4.1}
	\]


	непрерывны на интервале \( (a, b) \).

	Пусть \( \left\{ \varphi_k(x) \right\}_{k=1}^{n} \) — ФСР этого уравнения.

	Тогда общее решение уравнения имеет вид:


	\[
	y(x) = c_1 \varphi_1(x) + c_2 \varphi_2(x) + \ldots + c_n \varphi_n(x),
	\]


	где \( c_1, c_2, \ldots, c_n \) — произвольные постоянные.

	\subsection*{Доказательство}

	\begin{enumerate}
	\item Для любого набора чисел \( c_1, c_2, \ldots, c_n \), функция \( y(x) \) является решением уравнения, так как \( L \) — линейный оператор.

	\item Пусть \( \tilde{y}(x) \) — решение уравнения. Выберем произвольную точку \( x_0 \in (a, b) \) и вычислим значения:
	

	\[
	\tilde{y}(x_0) = y_0, \quad \tilde{y}'(x_0) = y_1, \quad \ldots, \quad \tilde{y}^{(n-1)}(x_0) = y_{n-1}.
	\]


	\end{enumerate}

	Рассмотрим СЛАУ:



	\[
	\begin{cases}
	c_1 \varphi_1(x_0) + c_2 \varphi_2(x_0) + \ldots + c_n \varphi_n(x_0) = y_0 \\
	c_1 \varphi_1'(x_0) + c_2 \varphi_2'(x_0) + \ldots + c_n \varphi_n'(x_0) = y_1 \\
	\vdots \\
	c_1 \varphi_1^{(n-1)}(x_0) + c_2 \varphi_2^{(n-1)}(x_0) + \ldots + c_n \varphi_n^{(n-1)}(x_0) = y_{n-1}
	\end{cases}
	\]



	Её определитель \( W(x_0) \neq 0 \).

	Следовательно, существует решение \( c_1^*, c_2^*, \ldots, c_n^* \) этой системы.

	Функция


	\[
	z(x) = c^*_1 \varphi_1(x) + c^*_2 \varphi_2(x) + \ldots + c^*_n \varphi_n(x)
	\]


	является решением однородного ЛДУ (4.1), так как \( L \) — линейный оператор.

	Функция \( z(x) \) удовлетворяет начальным условиям:


	\[
	z(x_0) = y_0, \quad z'(x_0) = y_1, \quad \ldots, \quad z^{(n-1)}(x_0) = y_{n-1}.
	\]



	Следовательно, по теореме единственности, эти решения совпадают, то есть


	\[
	\widetilde{y}(x) = c^*_1 \varphi_1(x) + c^*_2 \varphi_2(x) + \ldots + c^*_n \varphi_n(x) \quad \text{на } (a, b).
	\]



	Таким образом, для любого решения \( \widetilde{y}(x) \) существуют такие числа \( c^*_1, c^*_2, \ldots, c^*_n \), что


	\[
	\widetilde{y}(x) = c^*_1 \varphi_1(x) + c^*_2 \varphi_2(x) + \ldots + c^*_n \varphi_n(x).
	\]

	\subsection*{Следствие}

	Фундаментальная система решений (ФСР) является базисом пространства решений однородного линейного дифференциального уравнения (ЛДУ), то есть базисом \( \ker L \).

	Размерность пространства решений однородного ЛДУ равна \( n \), то есть


	\[
	n = \dim \ker L.
	\]

	\newpage
}