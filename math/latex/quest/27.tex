{
	\section{Интегральный признак сходимости рядов. Сходимость обобщенного гармонического ряда.}

	\section*{Теорема 2.7 (Интегральный признак сходимости ряда).}

	Пусть функция \( f(x) \) определена на промежутке \( [1, +\infty) \), причём


	\[
	f(x) \geq 0 \quad \text{и} \quad f(x) \text{ монотонно убывает на } [1, +\infty).
	\]



	Тогда ряд


	\[
	\sum_{n=1}^{\infty} f(n)
	\]


	сходится тогда и только тогда, когда сходится интеграл


	\[
	\int_{1}^{+\infty} f(x) \, dx.
	\]

\section* {Следствия.}

\begin{enumerate}
  \item Утверждение теоремы справедливо и в случае, когда индекс суммирования начинается с \( p > 1 \); в этом случае нижний предел интеграла также меняется на \( p \).
  
  \item Обобщённый гармонический ряд
  

\[
  \sum_{n=1}^{\infty} \frac{1}{n^p}
  \]


  сходится при \( p > 1 \), расходится при \( p \leq 1 \).
\end{enumerate}



	\newpage
}