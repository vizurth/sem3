{
	\section{Формулировка теоремы о существовании и единственности решения задачи Коши для нормального ДУ n-го
	порядка.}

	\subsection*{Определение 3.1}

	ДУ \( n \)-го порядка, разрешенное относительно старшей производной, называется нормальным.



	\[
	y^{(n)} = f(x, y, y', \ldots, y^{(n-1)})
	\]

	\subsection*{Теорема 3.1 — Существование и единственность решения задачи Коши для нормального ДУ \( n \)-го порядка}

	Пусть \( f \) непрерывна и имеет непрерывные частные производные по 2-й, 3-й, ..., \( n+1 \)-й переменным в окрестности некоторой точки \( (x_0, y_0, y_1, \ldots, y_{n-1}) \). Тогда существует интервал \( [x_0 - h, x_0 + h] \) и определённая на нём \( n \) раз дифференцируемая функция \( y(x) \), которая удовлетворяет уравнению и начальным условиям:



	\[
	y^{(n)} = f(x, y, y', \ldots, y^{(n-1)}), \quad y(x_0) = y_0, \quad y'(x_0) = y_1, \quad \ldots, \quad y^{(n-1)}(x_0) = y_{n-1}.
	\]



Такая функция единственна (без доказательства).



	\newpage
}