{
	\section{Теорема о дифференцировании и интегрировании суммы степенного ряда. Бесконечная дифференцируемость
	суммы степенного ряда. Связь коэффициентов степенного ряда с производными его суммы.}

	\section*{Теорема 6.3} 
	Пусть \( R \) — радиус сходимости степенного ряда \[ \sum_{n=0}^{\infty} a_n x^n \tag{6.1} \] и пусть \( u(x) \) — его сумма. Тогда: 
	\begin{enumerate} 
		\item Для всех \( x \in (-R, R) \) степенной ряд \[ \sum_{n=0}^{\infty} \frac{a_n x^{n+1}}{n+1} \] сходится к \[ \int_0^x u(t) \, dt. \] \item Для всех \( x \in (-R, R) \) степенной ряд \[ \sum_{n=1}^{\infty} n a_n x^{n-1} \] сходится к \( u'(x) \).
	\end{enumerate}

\section*{Следствие}

Степенной ряд можно почленно интегрировать и дифференцировать; при этом его радиус сходимости не изменится.

\section*{Теорема 6.4}

Пусть \( R \) — радиус сходимости степенного ряда


\[
\sum_{n=0}^{\infty} a_n x^n \tag{6.1}
\]


и пусть \( u(x) \) — его сумма. Тогда на интервале \( (-R, R) \) функция \( u(x) \) имеет производные всех порядков, причём


\[
u^{(k)}(x) = \sum_{n=k}^{\infty} n(n-1)\ldots(n-k+1) a_n x^{n-k}.
\]



\section*{Следствия}

\begin{enumerate} \item Для всех \( k \in \mathbb{N} \): \[ u^{(k)}(0) = k! \cdot a_k \quad \Rightarrow \quad a_k = \frac{u^{(k)}(0)}{k!}. \] Если \[ u(x) = \sum_{n=0}^{\infty} a_n (x - x_0)^n, \] то \[ u^{(k)}(x_0) = k! \cdot a_k \quad \Rightarrow \quad a_k = \frac{u^{(k)}(x_0)}{k!}. \] Следовательно, коэффициенты степенного ряда полностью определяются значениями функции \( u(x) \) и её производных в точке \( x_0 \). \item Если существует разложение функции в степенной ряд по степеням \( (x - x_0) \), то это разложение единственно. \end{enumerate}


	\newpage
}