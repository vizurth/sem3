{
	\section{Признак Коши.}

	\subsection*{Теорема 2.5 (Признак Коши).}

Рассмотрим ряд


\[
\sum_{n=1}^{\infty} a_n
\]


где \( a_n > 0 \quad \forall n \in \mathbb{N} \).

\begin{enumerate}
  \item Если \( \exists\, q < 1 : \sqrt[n]{a_n} \leq q \quad \forall n \in \mathbb{N} \), то ряд сходится.
  \item Если \( \exists\, q > 1 : \sqrt[n]{a_n} \geq q \quad \forall n \in \mathbb{N} \), то ряд расходится.
\end{enumerate}

\medskip


\subsection*{Теорема 2.6 (Предельный признак Коши).}

Рассмотрим ряд


\[
\sum_{n=1}^{\infty} a_n
\]


где \( a_n > 0 \quad \forall n \in \mathbb{N} \).

Пусть существует предел


\[
\lim_{n \to \infty} \sqrt[n]{a_n} = q.
\]



Тогда:
\begin{enumerate}
  \item если \( q < 1 \), ряд сходится;
  \item если \( q > 1 \), ряд расходится;
  \item если \( q = 1 \), ничего сказать нельзя.
\end{enumerate}

\medskip

\textit{(Доказательство аналогично теореме 2.4).}

\medskip

\subsection*{Следствие.}

В случае \( q > 1 \) общий член ряда \( a_n \) не стремится к нулю.

	\newpage
}