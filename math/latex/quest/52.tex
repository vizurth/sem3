{
	\section{Теорема о сходимости тригонометрического ряда Фурье кусочно-дифференцируемой на [-π, π] функции. Теорема
	о сходимости тригонометрического ряда Фурье 2π-периодической кусочно-дифференцируемой функции.}


\section*{Определение 10.2.}

Пусть $f$ — кусочно–непрерывная функция на $[-\pi,\pi]$, то есть,
имеет конечное число точек разрыва первого рода на $[-\pi,\pi]$.

Если

1) в каждой точке, где $f$ непрерывна, существуют либо $f'$, либо
$f'_+$ и $f'_-$;

2) в каждой точке $x_0$ разрыва $f$ существуют производные

а) левая для $f$, переопределенной в точке $x_0$ значением
$f(x_0-0)$,

б) правая для $f$, переопределенной в точке $x_0$ значением
$f(x_0+0)$;

то такие функции называются кусочно–дифференцируемыми.

\medskip

\section*{Теорема 10.2.}

Тригонометрический ряд Фурье кусочно–дифференцируемой на
$[-\pi,\pi]$ функции $f$ сходится к ней в каждой точке из
$[-\pi,\pi]$, где $f$ непрерывна.

В каждой точке $x_0$ разрыва функции $f$ ряд Фурье сходится к
\[
\frac{f(x_0-0) + f(x_0+0)}{2}.
\]

\[
(\text{без доказательства})
\]

\medskip

\section*{Теорема 10.3.}

Пусть $f$ — $2\pi$–периодичная кусочно–дифференцируемая функция
(то есть, удовлетворяет определению 10.2 на $[-\pi,\pi]$).
Тогда её тригонометрический ряд Фурье сходится к ней в каждой точке,
где $f$ непрерывна.

В каждой точке $x_0$ разрыва функции $f$ ряд Фурье сходится к
\[
\frac{f(x_0-0) + f(x_0+0)}{2}.
\]

\medskip

\section*{Лемма 10.2.}

Пусть $f$ имеет период $T$. Тогда $\forall\, a$
\[
\int_{0}^{T} f(x)\,dx
=
\int_{a}^{T+a} f(x)\,dx.
\]

(то есть, интегралы от $f$ по любому промежутку длиной в период
совпадают).



	\newpage
}