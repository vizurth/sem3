{
	\section{Теорема о существовании ФСР однородного ЛДУ. (Доказательство)}

	\subsection*{Теорема 4.2}

	Пусть все коэффициенты \( p_{n-1}(x), p_{n-2}(x), \ldots, p_1(x), p_0(x) \) уравнения


	\[
	y^{(n)} + p_{n-1}(x)y^{(n-1)} + \ldots + p_1(x)y' + p_0(x)y = 0 \tag{4.1}
	\]


	непрерывны на интервале \( (a, b) \).

	Тогда существует фундаментальная система решений (ФСР) этого уравнения на \( (a, b) \).

	\subsection*{Доказательство}

	Выберем произвольную точку \( x_0 \in (a, b) \) и рассмотрим следующие задачи Коши:



	\[
	\begin{cases}
	y(x_0) = 1 \\
	y'(x_0) = 0 \\
	y''(x_0) = 0 \\
	\vdots \\
	y^{(n-1)}(x_0) = 0
	\end{cases}
	\quad
	\begin{cases}
	y(x_0) = 0 \\
	y'(x_0) = 1 \\
	y''(x_0) = 0 \\
	\vdots \\
	y^{(n-1)}(x_0) = 0
	\end{cases}
	\quad
	\cdots
	\quad
	\begin{cases}
	y(x_0) = 0 \\
	y'(x_0) = 0 \\
	y''(x_0) = 0 \\
	\vdots \\
	y^{(n-1)}(x_0) = 1
	\end{cases}
	\]


	Их (задач Коши) \( n \) штук, и у каждой существует решение.

	Пусть \( \varphi_1(x) \) — решение первой задачи Коши, \( \varphi_2(x) \) — решение второй задачи Коши, \ldots, \( \varphi_n(x) \) — решение \( n \)-й задачи Коши. Тогда система функций \( \left\{ \varphi_k(x) \right\}_{k=1}^{n} \) является фундаментальной системой решений (ФСР), так как



	\[
	W(x) = |E| = 1 \,\, (\neq 0).
	\]

	\newpage
}