{
	\section{Метод вариации произвольных постоянных (метод Лагранжа) нахождения решения неоднородного ЛДУ.}

	\subsection*{Неоднородные ЛДУ}

	Общее неоднородное линейное дифференциальное уравнение имеет вид:
	\begin{equation}
	y^{(n)} + p_{n-1}(x)y^{(n-1)} + \ldots + p_1(x)y' + p_0(x)y = q(x) \tag{6.1}
	\end{equation}
	Эквивалентно:


	\[
	L(y) = q(x)
	\]


	где коэффициенты \( p_{n-1}(x), \ldots, p_1(x), p_0(x) \) и функция \( q(x) \) непрерывны на интервале \( (a, b) \).

	\section*{Метод вариации постоянных}

	Пусть \( \left\{ \varphi_k(x) \right\}_{k=1}^{n} \) — фундаментальная система решений соответствующего однородного ЛДУ. Будем искать решение неоднородного ЛДУ (6.1) в виде:


	\[
	y(x) = c_1(x)\varphi_1(x) + c_2(x)\varphi_2(x) + \ldots + c_n(x)\varphi_n(x) = \sum_{j=1}^{n} c_j(x)\varphi_j(x),
	\]


	где \( c_1(x), c_2(x), \ldots, c_n(x) \) — неизвестные функции, которые нужно найти.


Имеем:


\[
y'(x) = \sum_{j=1}^{n} c_j'(x) \varphi_j(x) + \sum_{j=1}^{n} c_j(x) \varphi_j'(x)
\]


Наложим условие:


\[
\sum_{j=1}^{n} c_j'(x) \varphi_j(x) = 0
\]


Тогда:


\[
y'(x) = \sum_{j=1}^{n} c_j(x) \varphi_j'(x)
\]



Аналогично:


\[
y''(x) = \sum_{j=1}^{n} c_j'(x) \varphi_j'(x) + \sum_{j=1}^{n} c_j(x) \varphi_j''(x)
\]


Наложим условие:


\[
\sum_{j=1}^{n} c_j'(x) \varphi_j'(x) = 0
\]


Тогда:


\[
y''(x) = \sum_{j=1}^{n} c_j(x) \varphi_j''(x)
\]



Продолжая по аналогии:


\[
y^{(3)}(x) = \sum_{j=1}^{n} c_j'(x) \varphi_j''(x) + \sum_{j=1}^{n} c_j(x) \varphi_j^{(3)}(x)
\]


Наложим условие:


\[
\sum_{j=1}^{n} c_j'(x) \varphi_j''(x) = 0
\]


Тогда:


\[
y^{(3)}(x) = \sum_{j=1}^{n} c_j(x) \varphi_j^{(3)}(x)
\]



...

И, наконец:


\[
y^{(n)}(x) = \sum_{j=1}^{n} c_j'(x) \varphi_j^{(n-1)}(x) + \sum_{j=1}^{n} с_j(x) \varphi_j^{(n)}(x)
\]


Наложим условие:


\[
\sum_{j=1}^{n} c_j'(x) \varphi_j^{(n-1)}(x) = 0
\]


Тогда:


\[
y^{(n)}(x) = \sum_{j=1}^{n} c_j(x) \varphi_j^{(n)}(x)
\]



	\subsection*{Подстановка в левую часть уравнения}

	Подставим в левую часть уравнения:


	\[
	\sum_{j=1}^{n} c'_j(x) \varphi_j^{(n-1)}(x) + \sum_{j=1}^{n} c_j(x) \varphi_j^{(n)}(x) + p_{n-1}(x)\sum_{j=1}^{n} c_j(x) \varphi_j^{(n-1)}(x) + \ldots
	\]




	\[
	+ p_1(x)\sum_{j=1}^{n} c_j(x) \varphi_j'(x) + p_0(x)\sum_{j=1}^{n} c_j(x) \varphi_j(x) =
	\]




	\[
	= \sum_{j=1}^{n} c'_j(x) \varphi_j^{(n-1)}(x) + \sum_{j=1}^{n} c_j(x) L(\varphi_j(x)) = \sum_{j=1}^{n} c'_j(x) \varphi_j^{(n-1)}(x).
	\]
	(так как $L(\varphi_j(x)) = 0$).

	Следовательно $\sum_{j=1}^{n} c'_j(x) \varphi_j^{(n-1)}(x) = q(x)$







	\subsection*{Система линейных алгебраических уравнений}

	Получили систему линейных алгебраических уравнений (СЛАУ) относительно \( c'_1(x), c'_2(x), \ldots, c'_n(x) \):


	\[
	\sum_{j=1}^{r} c'_j(x) \varphi_j(x) = 0
	\]




	\[
	\sum_{j=1}^{r} c'_j(x) \varphi_j^{(1)}(x) = 0
	\]




	\[
	\sum_{j=1}^{r} c'_j(x) \varphi_j^{(2)}(x) = 0
	\]

	\(
		\ldots	
	\)

	\[
	\sum_{j=1}^{r} c'_j(x) \varphi_j^{(n-2)}(x) = 0
	\]


	\[
	\sum_{j=1}^{r} c'_j(x) \varphi_j^{(n-1)}(x) = q(x)
	\]

	Её определитель \( W(x) \neq 0 \) на интервале \( (a, b) \). Следовательно, существует единственное решение:


	\[
	c_1'(x) = f_1(x), \quad c_2'(x) = f_2(x), \quad \ldots, \quad c_n'(x) = f_n(x),
	\]


	где функции \( f_j(x) \) непрерывны на \( (a, b) \), так как выражаются через функции \( \varphi_1(x), \varphi_2(x), \ldots, \varphi_n(x) \) и их производные до порядка \( n-1 \) (вспомните формулы Крамера).

	Следовательно:


	\[
	c_1(x) = F_1(x) + c^*_1, \quad c_2(x) = F_2(x) + c^*_2, \quad \ldots, \quad c_n(x) = F_n(x) + c^*_n,
	\]


	где \( F_1(x), F_2(x), \ldots, F_n(x) \) — некоторые первообразные функции.

	Положим \( c^*_1 = c^*_2 = \ldots = c^*_n = 0 \), тогда получим конкретные функции \( c_1(x), c_2(x), \ldots, c_n(x) \) и конкретное (то есть частное) решение:


	\[
	y(x) = F_1(x)\varphi_1(x) + F_2(x)\varphi_2(x) + \ldots + F_n(x)\varphi_n(x)
	\]

	\subsection*{Замечание}

Тогда общее решение неоднородного ЛДУ (6.1) будет иметь вид:


\[
y(x) = F_1(x) + F_2(x) + \ldots + F_n(x) + c_1 \varphi_1(x) + c_2 \varphi_2(x) + \ldots + c_n \varphi_n(x),
\]


где \( c_1 \varphi_1(x) + c_2 \varphi_2(x) + \ldots + c_n \varphi_n(x) \) — общее решение соответствующего однородного ЛДУ.

Следовательно, если бы мы в выражениях


\[
c_1 F_1(x) + c_2 F_2(x) + \ldots + c_n F_n(x), \quad
c_1 \varphi_1(x) + c_2 \varphi_2(x) + \ldots + c_n \varphi_n(x)
\]


принимали произвольные постоянные \( c_1, c_2, \ldots, c_n \), то сразу бы получили общее решение неоднородного ЛДУ (6.1).


	\newpage
}