% {
% 	\section{Нормальные системы дифференциальных уравнений (СДУ). Запись системы в векторной форме. Определение
% решения. Теорема о существовании и единственности решения задачи Коши для нормальных систем.}

% \subsection*{Нормальная система ОДУ}

% Система \textbf{(7.1)} $n$ дифференциальных уравнений первого порядка называется \textbf{нормальной системой}, если она имеет вид:


% \[
% \begin{cases}
% y_1' = f_1(x, y_1, \ldots, y_n), \\
% y_2' = f_2(x, y_1, \ldots, y_n), \\
% \vdots \\
% y_n' = f_n(x, y_1, \ldots, y_n),
% \end{cases}
% \quad \iff \quad
% Y'(x) = F(x, Y),
% \]


% где


% \[
% Y(x) =
% \begin{pmatrix}
% y_1(x) \\
% y_2(x) \\
% \vdots \\
% y_n(x)
% \end{pmatrix},
% \quad
% F(x, Y) =
% \begin{pmatrix}
% f_1(x, y_1, \ldots, y_n) \\
% f_2(x, y_1, \ldots, y_n) \\
% \vdots \\
% f_n(x, y_1, \ldots, y_n)
% \end{pmatrix}.
% \]
% вектор функции ,  называется нормальной системой    ОДУ   n-го порядка   ( порядок уравнения –количество неизвестных функций    y1(x) ,  y2(x) , ... , yn(x) ,   производные  участвуют только первого порядка ) .

% \[
% \text{(Считаем, что }
% Y'(x) =
% \begin{pmatrix}
% y_1'(x) \\
% y_2'(x) \\
% \vdots \\
% y_n'(x)
% \end{pmatrix}
% \text{)}
% \]


% \subsection*{Определение решения}

% Пусть функции $f_1, f_2, \ldots, f_n$ определены и непрерывны на множестве $\Omega \subset \mathbb{R}^{n+1}$. Набор функций $y_1(x), y_2(x), \ldots, y_n(x)$, определённых на некотором промежутке $(a, b)$, называется \textbf{решением системы} (7.1), если выполняются следующие условия:
% \begin{enumerate}
%     \item $(x, y_1(x), y_2(x), \ldots, y_n(x)) \in \Omega \quad \forall x \in (a, b);$
%     \item $\exists \, y_1', y_2', \ldots, y_n' \text{ на } (a, b);$
%     \item после подстановки функций $y_1(x), y_2(x), \ldots, y_n(x)$ в систему (7.1) получаются верные тождества на $(a, b)$.
% \end{enumerate}


% \subsection*{Теорема о существовании и единственности решения задачи Коши}

% Пусть функции $f_1, f_2, \ldots, f_n$ определены и непрерывны на множестве $\Omega \subset \mathbb{R}^{n+1}$, и все элементы матрицы Якоби


% \[
% \begin{pmatrix}
% \frac{\partial f_1}{\partial y_1} & \frac{\partial f_1}{\partial y_2} & \cdots & \frac{\partial f_1}{\partial y_n} \\
% \frac{\partial f_2}{\partial y_1} & \frac{\partial f_2}{\partial y_2} & \cdots & \frac{\partial f_2}{\partial y_n} \\
% \vdots & \vdots & \ddots & \vdots \\
% \frac{\partial f_n}{\partial y_1} & \frac{\partial f_n}{\partial y_2} & \cdots & \frac{\partial f_n}{\partial y_n}
% \end{pmatrix}
% \]


% непрерывны на множестве $\Omega$.

% Тогда для любого набора чисел $(x_0, y_1^*, \ldots, y_n^*) \in \Omega$ существует единственное решение $y_1(x), y_2(x), \ldots, y_n(x)$ системы (7.1), определённое на некотором интервале $[x_0 - h, x_0 + h]$ и удовлетворяющее начальным условиям:


% \[
% y_1(x_0) = y_1^*, \quad y_2(x_0) = y_2^*, \quad \ldots, \quad y_n(x_0) = y_n^*.
% \]


% \textit{(без доказательства)}


% 	\newpage
% }