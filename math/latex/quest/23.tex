{
	\section{Остаток ряда. Связь между сходимостью ряда и его остатка. Необходимое условие сходимости ряда.}

	\subsection*{Определение 1.2.}

	\[
	\sum_{n=1}^{\infty} a_n \quad \text{(1)}
	\]



	Ряд


	\[
	\sum_{n=m+1}^{\infty} a_n = a_{m+1} + a_{m+2} + \ldots \quad \text{(2)}
	\]


	называется \textit{остатком ряда} (1) после \( m \)-го члена, сумму остатка обозначим \( r_m \).

	\vspace{1em}
	\subsection*{Теорема 1.3.}

	Ряд и его остаток сходятся или расходятся одновременно.

	\subsection*{Следствия}

	\begin{enumerate}
		\item Если из ряда изъять (или добавить) конечное число слагаемых, на сходимость это не повлияет.

		\item Ряд сходится тогда и только тогда, когда сумма остатка ряда после \( m \)-го члена \( r_m \to 0 \) при \( m \to \infty \), так как
		

	\[
		\sum_{n=1}^{\infty} a_n
		\]


		\begin{itemize}
			\item[а)] если ряд (1) сходится к \( S^{(1)} \), то
			

	\[
			S^{(1)} = S_m^{(1)} + r_m,
			\]


			так как \( S_m^{(1)} \to S^{(1)} \) при \( m \to \infty \), то \( r_m \to 0 \) при \( m \to \infty \);

			\item[б)] если сумма остатка ряда после \( m \)-го члена \( r_m \to 0 \) при \( m \to \infty \), следовательно, сумма остатка ряда после \( m \)-го члена существует при всех \( m \in \mathbb{N} \), и, следовательно, ряд сходится.
		\end{itemize}
	\end{enumerate}

	\subsection*{Теорема 1.4. Необходимый признак сходимости ряда.}

	Пусть ряд


	\[
	\sum_{n=1}^{\infty} a_n
	\]


	сходится. Тогда его общий член \( a_n \) стремится к нулю при \( n \to \infty \).

	\newpage
}