{
	\section{Теорема о структуре общего решения неоднородного ЛДУ. (Доказательство)}


	\subsection*{Неоднородные ЛДУ}

	Общее неоднородное линейное дифференциальное уравнение имеет вид:
	\begin{equation}
	y^{(n)} + p_{n-1}(x)y^{(n-1)} + \ldots + p_1(x)y' + p_0(x)y = q(x) \tag{6.1}
	\end{equation}
	Эквивалентно:


	\[
	L(y) = q(x)
	\]


	где коэффициенты \( p_{n-1}(x), \ldots, p_1(x), p_0(x) \) и функция \( q(x) \) непрерывны на интервале \( (a, b) \).



	\subsection*{Теорема 6.1}

	Общее решение неоднородного линейного дифференциального уравнения (ЛДУ) есть сумма частного решения неоднородного ЛДУ и общего решения соответствующего однородного ЛДУ.

	Соответствующее однородное ЛДУ имеет вид:


	\[
	y^{(n)} + p_{n-1}(x)y^{(n-1)} + \ldots + p_1(x)y' + p_0(x)y = 0
	\]

	\subsection*{Доказательство}

	Докажем, что


	\[
	y(x) = y_0(x) + c_1 \varphi_1(x) + c_2 \varphi_2(x) + \ldots + c_n \varphi_n(x),
	\]


	где \( y_0(x) \) — частное (то есть конкретное) решение неоднородного ЛДУ (6.1), а \( \left\{ \varphi_k(x) \right\}_{k=1}^{n} \) — ФСР соответствующего однородного ЛДУ, является общим решением уравнения (6.1).

	\begin{enumerate}
		\item Для любого набора чисел \( c_1, c_2, \ldots, c_n \), функция \( y(x) \) является решением уравнения (6.1), так как:
  

		\[
		  L(y(x)) = L(y_0(x)) + L(c_1\varphi_1(x) + c_2\varphi_2(x) + \ldots + c_n \varphi_n(x)) = q(x) + 0 = q(x)
		\]

		\item Пусть \( \tilde{y}(x) \) — решение уравнения (6.1). Рассмотрим функцию:
  

		\[
		  y(x) = \tilde{y}(x) - y_0(x)
		  \]
		
		
		  Тогда:
		  
		
		\[
		  L(y(x)) = L(\tilde{y}(x) - y_0(x)) = L(\tilde{y}(x)) - L(y_0(x)) = q(x) - q(x) = 0
		  \]
		


	\end{enumerate}

	Следовательно,


\[
y(x) = \tilde{y}(x) - y_0(x)
\]


— решение соответствующего однородного ЛДУ.

Следовательно, существуют числа \( \tilde{c}_1, \tilde{c}_2, \ldots, \tilde{c}_n \), такие что:


\[
y(x) = \tilde{y}(x) - y_0(x) = \tilde{c}_1 \varphi_1(x) + \tilde{c}_2 \varphi_2(x) + \ldots + \tilde{c}_n \varphi_n(x)
\]



Отсюда:


\[
\tilde{y}(x) = y_0(x) + \tilde{c}_1 \varphi_1(x) + \tilde{c}_2 \varphi_2(x) + \ldots + \tilde{c}_n \varphi_n(x)
\]



То есть для любого решения \( \tilde{y}(x) \) существуют числа \( \tilde{c}_1, \tilde{c}_2, \ldots, \tilde{c}_n \), такие что:


\[
\tilde{y}(x) = y_0(x) + \tilde{c}_1 \varphi_1(x) + \tilde{c}_2 \varphi_2(x) + \ldots + \tilde{c}_n \varphi_n(x)
\]

	\newpage
}