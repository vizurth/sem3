{
	\section{Линейные нормальные системы дифференциальных уравнений (СЛДУ). Запись в векторной форме. Теорема о
	существовании и единственности решения задачи Коши. Определение общего решения.}

	\section*{Определение системы линейных дифференциальных уравнений (СЛДУ)}

Система (7.2):


\[
\begin{cases}
y_1'(x) = a_{11}(x)y_1(x) + a_{12}(x)y_2(x) + \ldots + a_{1n}(x)y_n(x) + b_1(x), \\
y_2'(x) = a_{21}(x)y_1(x) + a_{22}(x)y_2(x) + \ldots + a_{2n}(x)y_n(x) + b_2(x), \\
\vdots \\
y_n'(x) = a_{n1}(x)y_1(x) + a_{n2}(x)y_2(x) + \ldots + a_{nn}(x)y_n(x) + b_n(x)
\end{cases}
\quad \iff \quad
Y'(x) = A(x)Y(x) + B(x)
\]



где


\[
A(x) =
\begin{pmatrix}
a_{11}(x) & a_{12}(x) & \ldots & a_{1n}(x) \\
a_{21}(x) & a_{22}(x) & \ldots & a_{2n}(x) \\
\vdots & \vdots & \ddots & \vdots \\
a_{n1}(x) & a_{n2}(x) & \ldots & a_{nn}(x)
\end{pmatrix}
= \{a_{ij}(x)\}_{1 \leq i,j \leq n}
\quad \text{— матрица коэффициентов,}
\]





\[
B(x) =
\begin{pmatrix}
b_1(x) \\
b_2(x) \\
\vdots \\
b_n(x)
\end{pmatrix}
\quad \text{— столбец свободных членов или правых частей}
\]



Такая система называется \textbf{системой линейных дифференциальных уравнений} (СЛДУ). Если $B(x) = 0$, то система называется \textbf{однородной}.


\section*{Теорема 7.2 о существовании и единственности решения
задачи Коши}

Пусть $a_{ij}(x), b_j(x)$ непрерывны на $(a, b) \quad (1 \leq i, j \leq n)$. Тогда для любого набора чисел


\[
(x_0, y_1^*, y_2^*, \ldots, y_n^*) \in (a, b) \times \mathbb{R}^n
\]


существует единственное решение


\[
y_1(x), y_2(x), \ldots, y_n(x)
\]


системы (7.2), удовлетворяющее начальным условиям:


\[
y_1(x_0) = y_1^*, \quad y_2(x_0) = y_2^*, \quad \ldots, \quad y_n(x_0) = y_n^*.
\]


\textit{(Следует из теоремы 7.1)}

\subsection*{Замечание}

Можно показать, что для СЛДУ все решения определены на всём промежутке $(a, b)$ \textit{(без доказательства)}.

\section*{Определение 7.4 общего решения}



\[
\begin{cases}
y_1 = y_1(x, c_1, \ldots, c_n) \\
\vdots \\
y_n = y_n(x, c_1, \ldots, c_n)
\end{cases}
\]



$n$-параметрический набор функций называется \textbf{общим решением СЛДУ} $n$-го порядка, если



\[
Y(x) =
\begin{pmatrix}
y_1(x, c_1, \ldots, c_n) \\
y_2(x, c_1, \ldots, c_n) \\
\vdots \\
y_n(x, c_1, \ldots, c_n)
\end{pmatrix}
\]


— решение системы, причём выполняются условия:

\begin{enumerate}
    \item для любого набора чисел $c_1, c_2, \ldots, c_n$;
    \item для любого решения $\widetilde{Y}(x)$ существуют числа $\widetilde{c}_1, \widetilde{c}_2, \ldots, \widetilde{c}_n$ такие, что
    

\[
    \widetilde{Y}(x) =
    \begin{pmatrix}
    y_1(x, \widetilde{c}_1, \ldots, \widetilde{c}_n) \\
    y_2(x, \widetilde{c}_1, \ldots, \widetilde{c}_n) \\
    \vdots \\
    y_n(x, \widetilde{c}_1, \ldots, \widetilde{c}_n)
    \end{pmatrix}.
    \]


\end{enumerate}

\subsection*{Замечание}

Из теоремы 7.2 следует, что в общем решении должно быть ровно $n$ произвольных постоянных.


	\newpage
}