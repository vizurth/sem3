{
	\section{Комплекснозначные функции действительной переменной. Лемма о комплекснозначном решении однородного
	ЛДУ.}

	\subsection*{Определение 5.1}


	Отображение \( y(x) : (a, b) \rightarrow \mathbb{C} \), сопоставляющее каждой точке промежутка \( (a, b) \) некоторое комплексное число, называется \textbf{комплекснозначной функцией вещественной переменной}.

	Любую такую функцию можно записать в виде:


	\[
	y(x) = u(x) + iv(x),
	\]


	где \( u(x) \) и \( v(x) \) — вещественнозначные функции.

	Если \( y(x_0) = c + id \), то \( u(x_0) = c \), а \( v(x_0) = d \).

	\subsection*{Теорема 5.1}

	\begin{enumerate}
	\item Если \( y(x) = u(x) + i v(x) \), где \( u(x) \) и \( v(x) \) — вещественнозначные функции, то производная комплекснозначной функции имеет вид:
	

	\[
	y'(x) = u'(x) + i v'(x).
	\]



	\item Для комплекснозначных функций справедливы все формулы и правила дифференцирования, аналогично вещественным функциям. (Без доказательства)
	\end{enumerate}

	\vspace{1em}

	\subsection*{Лемма 5.1}

	Функция \( y(x) = u(x) + i v(x) \) является решением ЛД) (5.1) тогда и только тогда, когда функции \( u(x) \) и \( v(x) \) являются решениями ЛДУ (5.1).



	\newpage
}