{
	\setcounter{section}{54}
	\section{Определение площади плоской фигуры. Необходимое и достаточное условие измеримости плоской фигуры.
Следствие.}


\subsection*{Определение}


Рассмотрим множество $A \subset \mathbb{R}^2$.
Обозначим через $\rho(a,b)$ расстояние между точками $a$ и $b$ множества $A$.
\textbf{Диаметром множества} $A$ называется
\[
d(A) = \sup_{a,b \in A} \rho(a,b).
\]


Пусть $F$ — ограниченное связное множество из $\mathbb{R}^2$.
Множество $F$ называется \textbf{плоской фигурой}.

Рассмотрим всевозможные многоугольные фигуры $P$, целиком лежащие внутри $F$
\[
P \subset F \setminus \partial F,
\]
и назовем их \textbf{вложенными}.

Рассмотрим всевозможные многоугольные фигуры $Q$, целиком содержащие $F$
\[
F \subset Q \setminus \partial Q,
\]
и назовем их \textbf{объемлющими}.

Площади $S(P)$ вложенных фигур $P$ ограничены сверху, а площади $S(Q)$
объемлющих фигур $Q$ ограничены снизу.

Следовательно, существует
\[
\sup_{P \subset F} S(P),
\]
которую обозначим через $S_*(F)$ и назовем \textbf{внутренней площадью} фигуры $F$,
а также существует
\[
\inf_{Q \supset F} S(Q),
\]
которую обозначим через $S^*(F)$ и назовем \textbf{внешней площадью} фигуры $F$.

Очевидно, что
\[
S_*(F) \le S^*(F).
\]

Если
\[
S_*(F) = S^*(F),
\]
то это значение обозначим через $S(F)$ и назовем \textbf{площадью фигуры} $F$.

Если площадь $S(F)$ существует, то фигура $F$ называется
\textbf{квадрируемой} или \textbf{измеримой}.

\subsection*{Теорема (необходимое и достаточное условие квадрируемости)}

Фигура $F$ квадрируема тогда и только тогда, когда
\[
\forall \, \varepsilon > 0 \quad \exists \, P, Q :
\quad S(Q) - S(P) < \varepsilon.
\]

\subsection*{Следствие}

Фигура $F$ квадрируема тогда и только тогда, когда ее граница $\partial F$
имеет меру ноль, то есть
\[
S(\partial F) = 0.
\]


\newpage
}