{
	\setcounter{section}{36}
	\section{Степенные ряды. Теорема о существовании радиуса сходимости степенного ряда (с леммой).
	Теорема о непрерывности суммы степенного ряда на концах интервала сходимости.}


\section*{Определение 6.1}

Функциональный ряд вида


\[
\sum_{n=0}^{\infty} a_n (x - x_0)^n,
\]


где \( x_0, a_0, a_1, \ldots, a_n, \ldots \in \mathbb{R} \), называется \textit{степенным рядом}. Числа \( a_0, a_1, \ldots, a_n, \ldots \) называются \textit{коэффициентами степенного ряда}.

\section*{Замечание}

Ряды вида


\[
\sum_{n=0}^{\infty} a_n x^n \tag{6.1}
\]


будем рассматривать, так как любой степенной ряд сводится к ряду такого вида заменой \( t = x - x_0 \).

\section*{Лемма 6.1} Если степенной ряд \[ \sum_{n=0}^{\infty} a_n x^n \] сходится в точке \( x_0 \), то он сходится абсолютно в любой точке \( x \), для которой \( |x| < |x_0| \). 
\section*{Следствие} Если степенной ряд \[ \sum_{n=0}^{\infty} a_n x^n \] расходится в точке \( x_0 \), то он расходится в любой точке \( x \), такой что \( |x| > |x_0| \).

\section*{Теорема 6.1} 
Рассмотрим степенной ряд \[ \sum_{n=0}^{\infty} a_n x^n \tag{6.1} \] 
Существует число 
\( R \in [0, \infty] \), 
обладающее следующими свойствами: 
\begin{enumerate} 
	\item \forall \, \( x \in (-R, R) \) ряд сходится абсолютно. 
	\item \forall \, \( x \in (-\infty, -R) \cup (R, +\infty) \) ряд расходится. '
	\item \forall \, \( r \in (0, R) \) ряд сходится равномерно на отрезке \( [-r, r] \). 
\end{enumerate}

\section*{Определение 6.2} Промежуток \( \langle -R, R \rangle \) называется \textit{промежутком сходимости} степенного ряда (6.1); число \( R \) называется \textit{радиусом сходимости} степенного ряда (6.1). \section*{Замечание} Поведение ряда на концах промежутка сходимости может быть различным. \section*{Следствие 1 к теореме 6.1} Если \( |x| > R \), то общий член ряда \[ \sum_{n=0}^{\infty} a_n x^n \] не стремится к нулю. 

\section*{Следствие 2 к теореме 6.1}

Сумма степенного ряда непрерывна во внутренних точках промежутка сходимости.

\section*{Теорема 6.2} Пусть \( R \in (0, \infty) \) — радиус сходимости степенного ряда \[ \sum_{n=0}^{\infty} a_n x^n \tag{6.1} \] Если ряд сходится в точке \( R \) (или \( -R \)), то его сумма непрерывна в этой точке слева (или справа соответственно). 



\newpage
}