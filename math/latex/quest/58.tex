{
	\section{Определение двойного интеграла.}

\subsection*{Определение 2.1}

Пусть множество $G \subset \mathbb{R}^2$ квадрируемо.
Пусть функция $f(x,y)$ определена и ограничена на $G$.

Разобьем множество $G$ на конечное число квадрируемых частей
\[
G_k \quad (k = 1,2,\ldots,n),
\]
не имеющих общих внутренних точек, и составим сумму
\[
\sigma_\tau
=
\sum_{k=1}^{n}
f(\xi_k,\eta_k)\,\Delta S_k,
\]
где $\Delta S_k$ — площадь множества $G_k$,
а $(\xi_k,\eta_k)$ — некоторая точка из $G_k$.

Такая сумма называется \textbf{интегральной суммой Римана},
соответствующей разбиению
\[
\tau = \{G_k\}_{k=1}^{n}
\]
множества $G$.

\subsection*{Определение 2.2}

\begin{enumerate}
    \item Число
    \[
    \lambda_\tau = \max\{ d(G_1), d(G_2), \ldots, d(G_n) \}
    \]
    называется \textbf{рангом разбиения} $\tau$.

    \item Если существует конечный предел интегральных сумм
    при стремлении ранга разбиения к нулю, и этот предел
    не зависит ни от выбора разбиения $\tau$,
    ни от выбора точек $(\xi_k,\eta_k)$,
    то этот предел называется \textbf{двойным интегралом}
    от функции $f(x,y)$ по множеству $G$ и обозначается
    \[
    \iint\limits_G f(x,y)\,dx\,dy.
    \]
\end{enumerate}

То есть,
\[
I = \lim_{\lambda_\tau \to 0} \sigma_\tau,
\]
что эквивалентно условию:
\[
\forall \varepsilon > 0 \ \exists \delta > 0 :
\bigl(
\lambda_\tau < \delta \Rightarrow
|\sigma_\tau - I| < \varepsilon
\bigr)
\]
для любого разбиения $\tau$ и любого выбора точек
$(\xi_k,\eta_k) \in G_k$.


\newpage
}