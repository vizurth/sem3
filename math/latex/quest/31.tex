{
	\section{Теорема о произведении абсолютно сходящихся рядов.}


\section*{Определение 3.3}

Рассмотрим ряды


\[
(1)\quad \sum_{n=1}^{\infty} a_n \quad \text{и} \quad (2)\quad \sum_{n=1}^{\infty} b_n.
\]


Ряд


\[
(3)\quad \sum_{n=1}^{\infty} c_n, \quad \text{где } c_n = a_1 b_n + a_2 b_{n-1} + \ldots + a_n b_1,
\]


называется произведением рядов (1) и (2).

\section*{Теорема 3.4}

Пусть ряды


\[
\sum_{n=1}^{\infty} a_n \quad \text{и} \quad \sum_{n=1}^{\infty} b_n
\]


сходятся абсолютно, и их суммы равны \( A \) и \( B \) соответственно. Тогда ряд \[ \sum_{n=1}^{\infty} c_n, \quad \text{где } c_n = a_1 b_n + a_2 b_{n-1} + \ldots + a_n b_1, \] также сходится абсолютно, и его сумма равна \( AB \).

	\newpage
}