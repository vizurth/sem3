{
	\section{Определение решения ОДУ. Эквивалентные дифференциальные уравнения. Задача Коши для ОДУ n-го порядка.}
	\subsection*{Определение 1.1}

	Уравнение


	\[
	F(x, y, y', y'', \ldots, y^{(n)}) = 0,
	\]


	где \( F \) — известная функция \( n + 2 \) переменных, \( x \) — независимая переменная, а \( y \) — функция, которую нужно найти, называется \textbf{обыкновенным дифференциальным уравнением (ОДУ) \( n \)-го порядка}.

	Функция \( y(x) \) называется \textbf{решением} уравнения.

	\subsection*{Определение 1.2}

	Пусть \( F(T) = F(t_1, t_2, \ldots, t_{n+2}) \) определена и непрерывна на множестве \( \Omega \subset \mathbb{R}^{n+2} \). Функция \( y(x) \), определённая на некотором промежутке \( (a, b) \), называется \textbf{решением ОДУ} (1), если выполняются условия:
	\begin{enumerate}
		\item $\exists\, y, \ldots, y^{(n)}$ на $(a, b),$
    	\item $(x, y, y', \ldots, y^{(n)}) \in \Omega \quad \forall\, x \in (a, b),$
    	\item $F(x, y, y', \ldots, y^{(n)}) = 0 \quad \forall\, x \in (a, b).$
	\end{enumerate}

	\subsection*{Пример}

	\begin{enumerate}
		\item $y' = -x y^2$ \quad (здесь $F(t, t_2, t_3) = t_1 t_2^2 + t_3$),
		\item $y = \dfrac{2}{x^2}$ \quad ($x \in (-\infty, 0)$) и $y = \dfrac{2}{x^2}$ \quad ($x \in (0, +\infty)$) — разные решения.
	\end{enumerate}



	\subsection*{Определение 1.3}

	График решения ОДУ называется \textbf{интегральной кривой} этого уравнения.


	\subsection*{Определение 1.4}

	Два алгебраических уравнения $F_1(T) = 0$ и $F_2(T) = 0$ называются \textbf{эквивалентными} на множестве $\Omega \subset \mathbb{R}^{n+2}$, если множества их решений совпадают.

	Соответственно, два ДУ называются эквивалентными на множестве $\Omega$, если на $\Omega$ эквивалентны соответствующие им алгебраические уравнения.

	Множества решений эквивалентных ДУ совпадают.

	\subsection*{Определение 1.5}

	\textbf{Задача Коши} для ДУ $n$-го порядка:

	Требуется найти решение $y(x)$ ДУ (1), удовлетворяющее начальным условиям


	\[
	y(x_0) = y_0,\quad y'(x_0) = y_1,\quad \ldots,\quad y^{(n-1)}(x_0) = y_{n-1},\quad y^{(n)} = y_n,
	\]


	где $(x_0, y_0, y_1, \ldots, y_{n-1})$ — заданные значения.

	В частности, для ДУ 1-го порядка $F(x, y, y') = 0$ имеем одно условие $y(x_0) = y_0$. То есть, требуется найти интегральную кривую, проходящую через точку $(x_0, y_0)$.

	Задача Коши может иметь или не иметь решение.



	\newpage
}