{
	\section{Первый и второй признаки сравнения рядов с неотрицательными членами.}

	\subsection*{Теорема 2.1.}

	Для того, чтобы ряд с неотрицательными членами сходился, необходимо и достаточно, чтобы последовательность его частичных сумм была ограничена сверху.

	\vspace{1em}
	\subsection*{Доказательство.}

	Рассмотрим ряд


	\[
	\sum_{n=1}^{\infty} a_n, \quad a_n \geq 0 \quad \forall n \in \mathbb{N}.
	\]


	Обозначим \( S_n \) — \( n \)-я частичная сумма ряда.

	\begin{enumerate}
		\item \( S_{n+1} - S_n = a_{n+1} \geq 0 \). Следовательно, последовательность частичных сумм \( \{S_n\}_{n=1}^{\infty} \) возрастает. Если она ограничена сверху, то имеет конечный предел по аксиоме Больцано–Вейерштрасса, то есть ряд сходится.

		\item Если ряд сходится, то последовательность частичных сумм \( \{S_n\}_{n=1}^{\infty} \) имеет конечный предел, а значит, она ограничена (см. 1 семестр).
	\end{enumerate}

	\subsection*{Теорема 2.2. Признаки сравнения рядов с неотрицательными членами.}

	Рассмотрим ряды:


	\[
	\text{(1)} \quad \sum_{n=1}^{\infty} a_n, \qquad \text{(2)} \quad \sum_{n=1}^{\infty} b_n, \qquad a_n, b_n \geq 0 \quad \forall n \in \mathbb{N}.
	\]



	\begin{enumerate}
		\item \textbf{Первый признак сравнения.}

		Пусть \( a_n \geq b_n \geq 0 \quad \forall n \in \mathbb{N} \). Тогда:
		\begin{itemize}
			\item[а)] если ряд (1) сходится, то ряд (2) также сходится;
			\item[б)] если ряд (2) расходится, то ряд (1) также расходится.
		\end{itemize}

		\item \textbf{Второй признак сравнения.}

		Пусть \( a_n, b_n > 0 \quad \forall n \in \mathbb{N} \) и существует конечный предел
		

	\[
		\lim_{n \to \infty} \frac{a_n}{b_n},
		\]


		не равный нулю.
	\end{enumerate}

	\vspace{1em}
	(например, \( a_n \sim b_n \) при \( n \to \infty \)).\\
	Тогда ряды (1) и (2) сходятся или расходятся одновременно.

	\subsection*{Доказательство.}

Пусть \( S_n^{(1)}, S_n^{(2)} \) — частичные суммы рядов (1) и (2).

\begin{enumerate}
    \item
    \begin{itemize}
        \item[а)] Если ряд (1) сходится, то последовательность частичных сумм \( \{S_n^{(1)}\}_{n=1}^{\infty} \) ограничена сверху. Так как \( S_n^{(1)} \geq S_n^{(2)} \quad \forall n \in \mathbb{N} \), то последовательность \( \{S_n^{(2)}\}_{n=1}^{\infty} \) также ограничена сверху. Следовательно, ряд (2) сходится.

        \item[б)] От противного: пусть ряд (2) расходится, а ряд (1) сходится. Тогда по пункту а) ряд (2) должен сходиться. Противоречие.
    \end{itemize}

    \item
    \begin{itemize}
        \item[а)] Пусть ряд (2) сходится. Так как существует конечный предел
        

\[
        \lim_{n \to \infty} \frac{a_n}{b_n},
        \]


        то последовательность \( \left\{ \frac{a_n}{b_n} \right\}_{n=1}^{\infty} \) ограничена, то есть существует \( M > 0 \), такое что
        

\[
        \frac{a_n}{b_n} \leq M \quad \forall n \in \mathbb{N}.
        \]


        Тогда \( a_n \leq M b_n \). Так как ряд \( \sum_{n=1}^{\infty} M b_n \) сходится (см. следствие к теореме 1.2), то ряд (1) сходится по первому признаку сравнения.

        \item[б)] Пусть ряд (1) сходится. Так как существует конечный предел
        

\[
        \lim_{n \to \infty} \frac{b_n}{a_n} = \frac{1}{\lim_{n \to \infty} \frac{a_n}{b_n}},
        \]


        то последовательность \( \left\{ \frac{b_n}{a_n} \right\}_{n=1}^{\infty} \) ограничена, то есть существует \( M > 0 \), такое что
        

\[
        \frac{b_n}{a_n} \leq M \quad \forall n \in \mathbb{N}.
        \]


        Тогда \( b_n \leq M a_n \). Так как ряд \( \sum_{n=1}^{\infty} M a_n \) сходится (см. следствие к теореме 1.2), то ряд (2) сходится по первому признаку сравнения.
    \end{itemize}
\end{enumerate}

	\newpage
}