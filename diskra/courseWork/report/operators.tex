\section{Реализация операций над мультимножествами}

В данном разделе приводится описание реализации теоретико-множественных и арифметических операций над мультимножествами.

\subsection{Теоретико-множественные операции}

\subsubsection{Метод Union(const Multiset\& other)}

\textbf{Назначение:} Выполняет операцию объединения двух мультимножеств. Для каждого элемента универсума кратность в результате устанавливается равной максимальной из кратностей в исходных множествах.

\textbf{Вход:}
\begin{itemize}
    \item \texttt{const Multiset\& other}: Второе мультимножество для объединения.
\end{itemize}

\textbf{Выход:}
\begin{itemize}
    \item \texttt{Multiset}: Новое мультимножество — результат объединения.
\end{itemize}

\begin{lstlisting}[style=cstyle, caption={Реализация Union()}]
Multiset Multiset::Union(const Multiset& other) const {
    Multiset result;
    for (auto& [el, cntU] : Universum.getElements()) {
        int cntA = elements.count(el) ? elements.at(el) : 0;
        int cntB = other.elements.count(el) ? other.elements.at(el) : 0;
        result.elements[el] = max(cntA, cntB);
    }
    result.recount();
    return result;
}
\end{lstlisting}

\subsubsection{Метод Intersection(const Multiset\& other)}

\textbf{Назначение:} Реализует операцию пересечения мультимножеств. Кратность каждого элемента в результате равна минимуму из кратностей в исходных множествах.

\textbf{Вход:}
\begin{itemize}
    \item \texttt{const Multiset\& other}: Второе мультимножество для пересечения.
\end{itemize}

\textbf{Выход:}
\begin{itemize}
    \item \texttt{Multiset}: Новое мультимножество — результат пересечения.
\end{itemize}

\begin{lstlisting}[style=cstyle, caption={Реализация Intersection()}]
Multiset Multiset::Intersection(const Multiset& other) const {
    Multiset result;
    for (auto& [el, cntU] : Universum.getElements()) {
        int cntA = elements.count(el) ? elements.at(el) : 0;
        int cntB = other.elements.count(el) ? other.elements.at(el) : 0;
        result.elements[el] = min(cntA, cntB);
    }
    result.recount();
    return result;
}
\end{lstlisting}

\subsubsection{Метод Complement()}

\textbf{Назначение:} Вычисляет дополнение мультимножества относительно универсума. Кратность каждого элемента в результате равна разности между кратностью в универсуме и кратностью в исходном множестве.

\textbf{Вход:} Нет параметров (используется статический универсум).

\textbf{Выход:}
\begin{itemize}
    \item \texttt{Multiset}: Новое мультимножество — дополнение текущего.
\end{itemize}

\begin{lstlisting}[style=cstyle, caption={Реализация Complement()}]
Multiset Multiset::Complement() const {
    Multiset result;
    for (auto& [el, cntU] : Universum.elements) {
        int cntA = elements.count(el) ? elements.at(el) : 0;
        result.elements[el] = max(0, cntU - cntA);
    }
    result.recount();
    return result;
}
\end{lstlisting}

\subsubsection{Метод Diff(const Multiset\& other)}

\textbf{Назначение:} Реализует теоретико-множественную разность мультимножеств. Кратность элемента в результате равна разности кратностей, но не может быть отрицательной.

\textbf{Вход:}
\begin{itemize}
    \item \texttt{const Multiset\& other}: Вычитаемое мультимножество.
\end{itemize}

\textbf{Выход:}
\begin{itemize}
    \item \texttt{Multiset}: Новое мультимножество — результат разности.
\end{itemize}

\begin{lstlisting}[style=cstyle, caption={Реализация Diff()}]
Multiset Multiset::Diff(const Multiset& other) const {
    Multiset result, right;
	right = other.Complement();

	result = this->Intersection(right);

    result.recount();
    return result;
}
\end{lstlisting}

\subsubsection{Метод SimmDiff(const Multiset\& other)}

\textbf{Назначение:} Реализует операцию симметрической разности мультимножеств. Кратность элемента равна абсолютному значению разности кратностей.

\textbf{Вход:}
\begin{itemize}
    \item \texttt{const Multiset\& other}: Второе мультимножество для операции.
\end{itemize}

\textbf{Выход:}
\begin{itemize}
    \item \texttt{Multiset}: Новое мультимножество — результат симметрической разности.
\end{itemize}

\begin{lstlisting}[style=cstyle, caption={Реализация SimmDiff()}]
Multiset Multiset::SimmDiff(const Multiset& other) const {
    Multiset result, left, right;

	left = this->Union(other);
	right = this->Intersection(other);
	result = left.Diff(right);

	result.recount();
	return result;
}
\end{lstlisting}

\subsection{Арифметические операции}

В отличие от теоретико-множественных операций, арифметические операции выполняют вычисления с кратностями элементов, используя стандартные арифметические правила. Все операции ограничены сверху кратностью элементов в универсуме.

\subsubsection{Оператор + (арифметическая сумма)}

\textbf{Назначение:} Реализует арифметическое сложение мультимножеств. Кратность элемента в результате равна сумме кратностей, ограниченной сверху кратностью в универсуме.

\textbf{Вход:}
\begin{itemize}
    \item \texttt{const Multiset\& other}: Второе слагаемое.
\end{itemize}

\textbf{Выход:}
\begin{itemize}
    \item \texttt{Multiset}: Новое мультимножество — результат сложения.
\end{itemize}

\begin{lstlisting}[style=cstyle, caption={Реализация operator+()}]
Multiset Multiset::operator+(const Multiset& other) const {
    Multiset result;
    for (auto& [el, cntU] : Universum.getElements()) {
        int cntA = elements.count(el) ? elements.at(el) : 0;
        int cntB = other.elements.count(el) ? other.elements.at(el) : 0;
        result.elements[el] = min(cntA + cntB, cntU);
    }
    result.recount();
    return result;
}
\end{lstlisting}

\subsubsection{Оператор - (арифметическая разность)}

\textbf{Назначение:} Выполняет арифметическое вычитание мультимножеств. Кратность элемента равна разности кратностей, но не может быть отрицательной.

\textbf{Вход:}
\begin{itemize}
    \item \texttt{const Multiset\& other}: Вычитаемое.
\end{itemize}

\textbf{Выход:}
\begin{itemize}
    \item \texttt{Multiset}: Новое мультимножество — результат вычитания.
\end{itemize}

\begin{lstlisting}[style=cstyle, caption={Реализация operator-()}]
Multiset Multiset::operator-(const Multiset& other) const {
    Multiset result;
    for (auto& [el, cntU] : Universum.getElements()) {
        int cntA = elements.count(el) ? elements.at(el) : 0;
        int cntB = other.elements.count(el) ? other.elements.at(el) : 0;
        result.elements[el] = max(0, cntA - cntB);
    }
    result.recount();
    return result;
}
\end{lstlisting}

\subsubsection{Оператор * (арифметическое произведение)}

\textbf{Назначение:} Реализует арифметическое умножение мультимножеств. Кратность элемента равна произведению кратностей, ограниченному сверху кратностью в универсуме.

\textbf{Вход:}
\begin{itemize}
    \item \texttt{const Multiset\& other}: Второй множитель.
\end{itemize}

\textbf{Выход:}
\begin{itemize}
    \item \texttt{Multiset}: Новое мультимножество — результат умножения.
\end{itemize}

\begin{lstlisting}[style=cstyle, caption={Реализация operator*()}]
Multiset Multiset::operator*(const Multiset& other) const {
    Multiset result;
    for (auto& [el, cntU] : Universum.getElements()) {
        int cntA = elements.count(el) ? elements.at(el) : 0;
        int cntB = other.elements.count(el) ? other.elements.at(el) : 0;
        result.elements[el] = min(cntA * cntB, cntU);
    }
    result.recount();
    return result;
}
\end{lstlisting}

\subsubsection{Оператор / (арифметическое деление)}

\textbf{Назначение:} Выполняет целочисленное деление кратностей мультимножеств. Кратность элемента равна целой части от деления, ограниченной сверху кратностью в универсуме. При делении на ноль кратность устанавливается в 0.

\textbf{Вход:}
\begin{itemize}
    \item \texttt{const Multiset\& other}: Делитель.
\end{itemize}

\textbf{Выход:}
\begin{itemize}
    \item \texttt{Multiset}: Новое мультимножество — результат деления.
\end{itemize}

\begin{lstlisting}[style=cstyle, caption={Реализация operator/()}]
Multiset Multiset::operator/(const Multiset& other) const {
    Multiset result;
    for (auto& [el, cntU] : Universum.getElements()) {
        int cntA = elements.count(el) ? elements.at(el) : 0;
        int cntB = other.elements.count(el) ? other.elements.at(el) : 0;
        int div = (cntB > 0) ? (cntA / cntB) : 0;
        result.elements[el] = min(div, cntU);
    }
    result.recount();
    return result;
}
\end{lstlisting}

\subsection{Вспомогательный метод recount()}

\textbf{Назначение:} Пересчитывает общую мощность мультимножества путём суммирования всех кратностей элементов.

\textbf{Примечание:} Данный метод вызывается после каждой операции, изменяющей содержимое мультимножества, чтобы поддерживать корректное значение \texttt{totalCardinality}.

\textbf{Вход:}
\begin{itemize}
    \item \texttt{map<string, int> elements;}: Кратность элементов.
\end{itemize}

\textbf{Выход:}
\begin{itemize}
    \item \texttt{int totalCardinality}: Поле для хранения общей мощности множества.
\end{itemize}

\begin{lstlisting}[style=cstyle, caption={Реализация recount()}]
void recount() {
    totalCardinality = 0;
    for (auto& [_, cnt] : elements)
        totalCardinality += cnt;
}
\end{lstlisting}