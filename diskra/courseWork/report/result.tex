\addcontentsline{toc}{section}{Заключение}
\section*{Заключение}

В ходе выполнения данной лабораторной работы была успешно реализована программа на языке C++, предназначенная для работы с мультимножествами на основе бинарного кода Грея.

\subsection*{Основные результаты}

В ходе работы были получены следующие результаты. Реализован эффективный алгоритм генерации универсума мультимножеств на основе бинарного кода Грея, где каждый следующий элемент отличается от предыдущего изменением одного бита. Добавлены два варианта заполнения мультимножеств: ручной и автоматический. Реализованы все основные операции над мультимножествами. Множественные: объединение, пересечение, дополнение, разность, симметрическая разность. Арифметические: сложение, вычитание, умножение, деление. Создан устойчивый к ошибкам пользовательский интерфейс, который корректно обрабатывает неверный ввод.

\subsection*{Плюсы реализации} 

\begin{enumerate}
    \item \textbf{Использование статического универсума:} Применение статического поля класса \texttt{Multiset::Universum} позволяет всем экземплярам мультимножеств работать с одним универсумом, что соответствует математической модели и упрощает реализацию операций.
    
    \item \textbf{Перегрузка операторов:} Арифметические операций сделаны через перегрузку стандартных операторов C++ (\texttt{+}, \texttt{-}, \texttt{*}, \texttt{/} ).
\end{enumerate}

\subsection*{Минусы реализации}

\begin{enumerate}
    \item \textbf{Ограничение на разрядность:} При больших значениях n (например, больше 20) размер универсума растет экспоненциально \( 2^n \), что ограничивает скорость и увеличивает используемый обьем памяти.
    
    \item \textbf{Отсутствие сохранения данных:} Программа не поддерживает сохранение и загрузку мультимножеств из файлов, что ограничивает её применение для работы данными которые уже были использованны.
\end{enumerate}

\subsection*{Возможности масштабирования и улучшения}

Программа имеет хороший потенциал для мастабирования:

\begin{enumerate}
    \item \textbf{Сохранение и загрузка данных:} Добавление функционала сохранения мультимножеств в файлы JSON для хранения.

    \item \textbf{Создание сложных выражений:} Добавить возможность вычислять сложные выражения например \(A \cap C \cup B\)
    
    \item \textbf{Графический интерфейс:} Разработка GUI с визуализацией множеств и результатов операций, на основе библиотеки Qt.
\end{enumerate}