\section{Математическое описание}


\subsection{«Малая» и «большая» конечные арифметики}

 Множество $Z$ вместе с набором операций $\Sigma = \{\varphi_1, \ldots, \varphi_m\}$, $\varphi_i : Z^{n_i} \rightarrow Z$, где $n_i$ — арность операции $\varphi_i$, называется алгебраической структурой, универсальной алгеброй или просто алгеброй.

%  Алгебраическая структура — множество значений, над которыми определены операции и отношения.

Коммутативное кольцо с единицей — алгебраическая структура $\langle Z; +, * \rangle$, в которой выполняются следующие аксиомы:

\begin{enumerate}
  \item Ассоциативность сложения: $(a + b) + c = a + (b + c)$
  \item Существование нулевого элемента: $\exists 0 \in Z \; (\forall a \in Z : a + 0 = 0 + a = a)$
  \item Существование противоположного элемента: $\forall a \in Z \; \exists (-a) \in Z : a + (-a) = 0$
  \item Коммутативность сложения: $a + b = b + a$
  \item Ассоциативность умножения: $(a * b) * c = a * (b * c)$
  \item Дистрибутивность: $a * (b + c) = a * b + a * c$
  \item Коммутативность умножения: $a * b = b * a$
  \item Существование единичного элемента: $\exists 1 \in Z \; (\forall a \in Z : a * 1 = 1 * a = a)$
\end{enumerate}

«Малая» конечная арифметика — конечное коммутативное кольцо с единицей $\langle Z_{8}; +, * \rangle$, на котором определены действия вычитания и деления, причём деление определено частично (только для элементов, имеющих мультипликативную инверсию).

В данной работе используется кольцо $Z_{8} = \{a, b, c, d, e, f, g, h\}$ размера $i = 8$, где:

\begin{itemize}
  \item Нулевой элемент (нейтральный по сложению): a
  \item Единичный элемент (нейтральный по умножению): b
  \item Отношение порядка (правило инкремента «+1»):\\
  $a \rightarrow b \rightarrow c \rightarrow e \rightarrow d \rightarrow g \rightarrow f \rightarrow h$
\end{itemize}

«Большая» конечная арифметика — конечное коммутативное кольцо с единицей $\langle Z_i^n; +, * \rangle$, элементами которого являются слова длины до $n$ над алфавитом $Z_i$. Операции определены позиционно с учётом переноса разрядов. Деление определено с остатком.

В данной работе $n = 8$ (максимум 8 разрядов), поэтому итоговая структура имеет вид $\langle Z_{8}^8; +, * \rangle$.

Каждому символу сопоставлен индекс (позиционное значение):



\[
\begin{array}{c|cccccccc}
\text{Символ} & a & b & c & e & d & g & f & h \\
\hline
\text{Индекс} & 0 & 1 & 2 & 3 & 4 & 5 & 6 & 7  \\
\end{array}
\]

\subsection{Таблицы операций малой арифметики}

На рисунках \ref{fig:add_table} и \ref{fig:mul_table} представлены таблицы операций и переносов для сло-
жения и умножения соответственно

\begin{figure}[h]
    \centering
    \begin{minipage}{0.48\textwidth}
        \centering
        \includegraphics[width=\linewidth]{screenshots/add_table.jpg}
    \end{minipage}
    \hfill
    \begin{minipage}{0.48\textwidth}
        \centering
        \includegraphics[width=\linewidth]{screenshots/carry_add_table.jpg}
    \end{minipage}
    \caption{Таблица сложения и таблица переносов для малой арифметики}
	\label{fig:add_table}
\end{figure}

\begin{figure}[h]
    \centering
    \begin{minipage}{0.48\textwidth}
        \centering
        \includegraphics[width=\linewidth]{screenshots/mul_table.jpg}
    \end{minipage}
    \hfill
    \begin{minipage}{0.48\textwidth}
        \centering
        \includegraphics[width=\linewidth]{screenshots/carry_mul_table.jpg}
    \end{minipage}
    \caption{Таблица умножения и таблица переносов для малой арифметики}
	\label{fig:mul_table}
\end{figure}
