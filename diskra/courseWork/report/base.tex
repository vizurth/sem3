\section{Математическое описание}

\subsection{Множества}
Множество — это фундаментальное понятие в математике, представляющее собой неупорядоченную совокупность уникальных элементов. В отличие от мультимножеств, элементы множества различны и отличимы друг от друга. Иными словами, множество можно рассматривать как частный случай мультимножества, в котором кратность каждого элемента равна единице. Операции над множествами, такие как объединение, пересечение и разность, могут быть обобщены и для мультимножеств, с учётом количества повторений элементов.


\subsection{Мультимножества}
Мультимножество $\hat{X}$ (над множеством $X$) называется совокупность элементов множества $X$, в которую элемент $x_i$ входит с кратностью $a_i \ge 0$. Мультимножество обозначается одним из следующих способов:
\[
    \hat{X} = [x_1{^{a_1}},...,x_n{^{a_n}}] =  
    \langle x_1,...,x_n;...;x_n,...,x_n \rangle  =
    \langle a_1(x_1),...,a_n(x_n) \rangle.
\]

В данной работе элементы мультимножеств представлены в виде кодов Грея, где разрядность $n$ определяет размер универсума, равный $2^n$ элементов.

\subsection{Бинарный код Грея}
Бинарный код Грея (отражённый бинарный код) — это такая последовательность двоичных чисел, в которой два соседних числа отличаются друг от друга только одним разрядом. Для генерации бинарного кода Грея заданной разрядности $n$ используется формула:
\begin{equation}
    G(i) = i \oplus (i \gg 1)
\end{equation}
где $G(i)$ — $i$-е число в коде Грея, $i$ — обычное бинарное представление числа ($i = 0, 1, ..., 2^n - 1$), $\oplus$ — операция побитового исключающего «ИЛИ» (XOR), и $\gg 1$ — побитовый сдвиг вправо на один разряд.

\textbf{Пример:} Для $n = 3$ генерируется последовательность из $2^3 = 8$ кодов Грея:
\begin{center}
\begin{tabular}{c|c|c}
$i$ (десятичное) & $i$ (двоичное) & $G(i)$ (код Грея) \\
\hline
0 & 000 & 000 \\
1 & 001 & 001 \\
2 & 010 & 011 \\
3 & 011 & 010 \\
4 & 100 & 110 \\
5 & 101 & 111 \\
6 & 110 & 101 \\
7 & 111 & 100 \\
\end{tabular}
\end{center}

\subsection{Операции над мультимножествами}

\subsubsection{Теоретико-множественные операции}
Для мультимножеств определяются следующие логические операции:

\begin{enumerate}
    \item \textbf{Объединение ($A \cup B$):} Кратность элемента в результате равна максимуму из кратностей в исходных мультимножествах:
    \[
C = A \cup B = \{x \mid x \in A \lor x \in B\}
\]
\[
C = A \cup B = \left\{ \max(a_i (x_i), a_j (x_j)) \right\}, \quad a_i (x_i) \in A, \; a_j (x_j) \in B
\]

    \item \textbf{Пересечение ($A \cap B$):} Кратность элемента равна минимуму из кратностей:  
\[
C = A \cap B = \{ x \mid x \in A \land x \in B \}
\]
\[
C = A \cap B = \left\{ \min(a_i(x_i), a_j(x_j)) \right\}, \quad a_i(x_i) \in A, \; a_j(x_j) \in B
\]

    \item \textbf{Разность ($A \setminus B$):} Кратность элемента равна разности кратностей, но не может быть отрицательной:  
    \[
    C = A \setminus B = A \cap \overline{B} = \{x \mid x \in A \land x \notin B\}
    \]
    \[ C = A \setminus B = A \cap \overline{B} = {\min(a_i(x_i),a_j(x_j))}, \quad a_i(x_i) \in A; a_j(x_j) \in \overline{B} \]

    \item \textbf{Дополнение ($\overline{A}$):} Дополнение мультимножества $A$ относительно универсума $U$ задаётся как:
    \[
    C = \overline{A} = U \setminus A = \{ x \mid x \notin A \}
    \]
    \[
    C = \overline{A} = \left\{ \max(a_i (x_i) - a_j (x_j), 0) \right\}, \quad a_i(x_i) \in U, \; a_j (x_j) \in A
    \]
    
    \item \textbf{Симметрическая разность ($A \triangle B$):} Кратность элемента равна модулю разности кратностей:
   \[
    C = A \Delta B = (A \cup B) \setminus (A \cap B) = \{x \mid (x \in A \land x \notin B) \lor (x \notin A \land x \in B)\}
    \]
    \[
    C = A \Delta B = (A \cup B) \setminus (A \cap B) = (A \cup B) \cap (\overline{A} \cup \overline{B}) =
    \]
    \[
    = (A \cap \overline{A}) \cup (A \cap \overline{B}) \cup  (B \cap \overline{A}) \cup  (B \cap \overline{B}) =
    \]
    \[
    = (A \cap \overline{B}) \cup  (B \cap \overline{A}) = (A \setminus B) \cup  (B \setminus A)
    \]

    \[
    C = A \Delta B = (A \setminus B) \cup  (B \setminus A) = 
    \]
    \[
    = \max(a_i(x_i) - a_j(x_j), 0) + \max(a_j(x_j) - a_i(x_i), 0); \quad a_i(x_i) \in A; a_j(x_j) \in B
    \]


    \
\end{enumerate}

\subsubsection{Арифметические операции}
Эти операции используют арифметические правила для вычисления кратности результирующих элементов:

\begin{enumerate}
    \item \textbf{Арифметическая сумма ($A + B$):} Кратность элемента равна сумме кратностей, ограниченной сверху кратностью в универсуме:
    \[
    C = A + B = \{x \mid x \in A \lor x \in B\}
    \]
    \[
    C = A + B = \left\{ \min(a_i (x_i) + a_j (x_j), u_k (x_k)) \right\}, \quad a_i(x_i) \in A, \; a_j (x_j) \in B, \; u_k (x_k) \in U
    \]

    \item \textbf{Арифметическая разность ($A - B$):} Кратность элемента равна разности кратностей, но не может быть отрицательной:
   \[
    C = A - B = \{x \mid x \in A \land x \in B\}
    \]
    \[
    C = A - B = \left\{ \max(a_i (x_i) - a_j (x_j), 0) \right\}, \quad a_i (x_i) \in A, \; a_j(x_j) \in B
    \]

    \item \textbf{Арифметическое произведение ($A \times B$):} Кратность элемента равна произведению кратностей, ограниченному сверху кратностью в универсуме:
    \[
    C = A \times B = \{x \mid x \in A \land x \in B\}
    \]
    \[
    C = A \times B = \left\{ \min(a_i (x_i), a_j(x_j), u_k (x_k)) \right\}, \quad a_i (x_i) \in A, \; a_j (x_j) \in B, \; u_k x_k \in U
    \]

    \item \textbf{Арифметическое деление ($A \div B$):} Кратность элемента равна целой части от деления кратностей, ограниченной сверху кратностью в универсуме:
    \[
    C = A \div B = \begin{cases}
        \left\lfloor \frac{a_i (x_i)}{a_j (x_j)} \right\rfloor & \text{при } a_j (x_j) \neq 0 \\
        0 \cdot x_i, & \text{при } a_j (x_j) = 0
    \end{cases}
     \,\,\,\,\,a_i (x_i) \in A, \; a_j (x_j) \in B
    \]
\end{enumerate}

\textbf{Примечание:} Во всех операциях, если кратность элемента становится равной нулю, он исключается из результирующего мультимножества (не хранится явно).