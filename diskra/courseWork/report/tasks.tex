\section{Постановка задач}

Цель данной лабораторной работы заключается в практической реализации алгоритмов дискретной математики с использованием языка программирования C++.


\subsection*{В рамках работы необходимо:}

\begin{enumerate}
    \item Разработать алгоритм генерации универсума мультимножеств на основе бинарного кода Грея заданной разрядности.
    \item Спроектировать и реализовать структуру данных для хранения мультимножеств, обеспечивающую эффективную работу с элементами и их кратностями.
    \item Реализовать два способа формирования мультимножеств из универсума:
    \begin{itemize}
        \item ручной — с последовательным вводом элементов пользователем;
        \item автоматический — с указанием требуемой мощности мультимножества.
    \end{itemize}
    \item Запрограммировать основные операции над мультимножествами:
    \begin{itemize}
        \item теоретико-множественные — объединение, пересечение, дополнение, разность и симметрическую разность;
        \item арифметические — сложение, вычитание, умножение и деление.
    \end{itemize}
    \item Создать удобный и устойчивый к ошибкам интерфейс для взаимодействия пользователя с программой..
    \item Провести тестирование программы на различных сценариях использования, включая граничные случаи и некорректный ввод данных.
\end{enumerate}

Результатом работы станет функциональная программа, демонстрирующая генерацию универсума, заполнение мультимножеств и выполнение различных операций над ними, а также подробный отчёт, описывающий теоретические основы, особенности реализации и результаты тестирования разработанного программного обеспечения.