\section{Результаты работы программы}

Программа имеет консольный интерфейс, позволяющий пользователю выполнять все реализованные операции через меню. В данном разделе представлены примеры выполнения основных функций, демонстрирующие корректность работы программы.

\subsection{Главное меню программы}

При запуске программы пользователю предоставляется главное меню с следующими опциями (см. рис.~\ref{fig:main_menu}).

\begin{figure}[H]
    \centering
    \includegraphics[width=0.69\linewidth]{myPhoto/main_menu.jpg}
    \caption{Главное меню}
    \label{fig:main_menu}
\end{figure}

\subsection{Генерация универсума}

После выбора первой опции пользователь может указать разрядность кода Грея. На рисунке~\ref{fig:univer} показан пример генерации универсума с разрядностью $n = 3$, что приводит к созданию $2^3 = 8$ элементов.

\begin{figure}[H]
    \centering
    \includegraphics[width=0.50\linewidth]{myPhoto/univer.jpg}
    \caption{Генерация универсума с разрядностью 3}
    \label{fig:univer}
\end{figure}

Каждому коду Грея присваивается случайная кратность в диапазоне от 1 до 50, что определяет максимальное количество вхождений элемента в любое мультимножество на основе данного универсума.

\subsection{Заполнение множеств вручную}

При выборе опции ручного заполнения программа последовательно запрашивает кратность для каждого элемента универсума. Пользователь может указать значение от 0 до максимальной кратности элемента в универсуме (см. рис.~\ref{fig:fillbyhand}).

\begin{figure}[H]
    \centering
    \includegraphics[width=0.50\linewidth]{myPhoto/fillbyhand.jpg}
    \caption{Ручное заполнение множества}
    \label{fig:fillbyhand}
\end{figure}

\subsection{Автоматическое заполнение множеств}

Автоматическое заполнение позволяет создать мультимножество заданной мощности путём случайного выбора элементов из универсума (см. рис.~\ref{fig:fillauto}).

\begin{figure}[H]
    \centering
    \includegraphics[width=0.69\linewidth]{myPhoto/fillauto.jpg}
    \caption{Автоматическое заполнение множества}
    \label{fig:fillauto}
\end{figure}

\subsection{Выполнение операций над множествами}

После заполнения множеств A и B становится доступным меню операций. Рассмотрим примеры выполнения различных операций.

\subsubsection{Теоретико-множественные операции}

\textbf{1. Объединение $A \cup B$:}

Как видно на рисунке~\ref{fig:union}, для каждого кода выбрана максимальная кратность из двух множеств (например, для кода 001: max(12, 18) = 18).

\begin{figure}[H]
    \centering
    \includegraphics[width=0.50\linewidth]{myPhoto/union.jpg}
    \caption{Объединение множеств}
    \label{fig:union}
\end{figure}

\textbf{2. Пересечение $A \cap B$:}

Для каждого элемента выбрана минимальная кратность (например, для кода 001: min(12, 18) = 12), как показано на рисунке~\ref{fig:inter}.

\begin{figure}[H]
    \centering
    \includegraphics[width=0.50\linewidth]{myPhoto/inter.jpg}
    \caption{Пересечение множеств}
    \label{fig:inter}
\end{figure}

\textbf{3. Дополнение к A $(U \,\, \textbackslash \,\, A)$:}

Результат операции дополнения представлен на рисунке~\ref{fig:cont}.

\begin{figure}[H]
    \centering
    \includegraphics[width=0.50\linewidth]{myPhoto/cont.jpg}
    \caption{Дополнение к множеству A}
    \label{fig:cont}
\end{figure}

\textbf{4. Разность $(A \,\, \textbackslash \,\, B)$:}

Элементы, где кратность в B больше или равна кратности в A, имеют нулевую кратность в результате (см. рис.~\ref{fig:diff}).

\begin{figure}[H]
    \centering
    \includegraphics[width=0.50\linewidth]{myPhoto/diff.jpg}
    \caption{Разность множеств}
    \label{fig:diff}
\end{figure}

\textbf{5. Симметрическая разность $(A \Delta B)$:}

Кратность каждого элемента равна |кратность в A - кратность в B|, как показано на рисунке~\ref{fig:simmdiff}.

\begin{figure}[H]
    \centering
    \includegraphics[width=0.50\linewidth]{myPhoto/simmdiff.jpg}
    \caption{Симметрическая разность множеств}
    \label{fig:simmdiff}
\end{figure}

\subsubsection{Арифметические операции}

\textbf{1. Арифметическая сумма $(A + B)$:}

Кратности складываются, но ограничиваются сверху кратностью в универсуме (например, для 111: min(2+7, 8) = 8), как показано на рисунке~\ref{fig:sum}.

\begin{figure}[H]
    \centering
    \includegraphics[width=0.50\linewidth]{myPhoto/sum.jpg}
    \caption{Арифметическая сумма множеств}
    \label{fig:sum}
\end{figure}

\textbf{2. Арифметическая разность $(A - B)$:}

Результат совпадает с теоретико-множественной разностью, так как обе операции используют формулу max(a - b, 0), что демонстрирует рисунок~\ref{fig:arfdiff}.

\begin{figure}[H]
    \centering
    \includegraphics[width=0.50\linewidth]{myPhoto/arfdidd.jpg}
    \caption{Арифметическая разность множеств}
    \label{fig:arfdiff}
\end{figure}

\textbf{3. Арифметическое произведение $(A * B)$:}

Кратности перемножаются и ограничиваются сверху универсумом, как показано на рисунке~\ref{fig:mult}.

\begin{figure}[H]
    \centering
    \includegraphics[width=0.50\linewidth]{myPhoto/mult.jpg}
    \caption{Арифметическое произведение множеств}
    \label{fig:mult}
\end{figure}

\textbf{4. Арифметическое деление $(A / B)$:}

Выполняется целочисленное деление кратностей (например, для 000: 8 / 5 = 1, для 110: 10 / 4 = 2), результат представлен на рисунке~\ref{fig:div}.

\begin{figure}[H]
    \centering
    \includegraphics[width=0.50\linewidth]{myPhoto/del.jpg}
    \caption{Арифметическое деление множеств}
    \label{fig:div}
\end{figure}

\subsection{Обработка некорректного ввода}

Программа включает механизмы обработки ошибок для обеспечения устойчивости к некорректному вводу.

\textbf{Пример 1: Ввод отрицательной разрядности}

На рисунке~\ref{fig:error1} показана реакция программы на ввод отрицательного значения разрядности.

\begin{figure}[H]
    \centering
    \includegraphics[width=0.50\linewidth]{myPhoto/first.jpg}
    \caption{Обработка ввода отрицательной разрядности}
    \label{fig:error1}
\end{figure}

\textbf{Пример 2: Ввод нечислового значения}

Рисунок~\ref{fig:error2} демонстрирует обработку некорректного (нечислового) ввода при указании мощности множества.

\begin{figure}[H]
    \centering
    \includegraphics[width=0.50\linewidth]{myPhoto/second.jpg}
    \caption{Обработка ввода нечислового значения}
    \label{fig:error2}
\end{figure}

\textbf{Пример 3: Превышение максимальной кратности при ручном заполнении}

На рисунке~\ref{fig:error3} показано, как программа обрабатывает попытку ввода кратности, превышающей максимально допустимое значение.

\begin{figure}[H]
    \centering
    \includegraphics[width=0.50\linewidth]{myPhoto/third.jpg}
    \caption{Обработка превышения максимальной кратности}
    \label{fig:error3}
\end{figure}

\textbf{Пример 4: Попытка выполнить операции без генерации универсума}

Рисунок~\ref{fig:error4} демонстрирует предупреждение программы при попытке работы с пустым универсумом.

\begin{figure}[H]
    \centering
    \includegraphics[width=0.50\linewidth]{myPhoto/fourth.jpg}
    \caption{Предупреждение о пустом универсуме}
    \label{fig:error4}
\end{figure}

Эти примеры демонстрируют, что программа корректно обрабатывает различные виды некорректного ввода и позволяет пользователю исправить ошибку без завершения работы программы.

\subsection{Проверка граничных случаев}

\textbf{Генерация пустого универсума (n = 0):}

На рисунке~\ref{fig:edge1} показана генерация универсума с нулевой разрядностью.

\begin{figure}[H]
    \centering
    \includegraphics[width=0.50\linewidth]{myPhoto/fifth.jpg}
    \caption{Генерация пустого универсума}
    \label{fig:edge1}
\end{figure}

\textbf{Автоматическое заполнение с нулевой мощностью:}

Рисунок~\ref{fig:edge2} демонстрирует создание пустого множества при указании нулевой мощности.

\begin{figure}[H]
    \centering
    \includegraphics[width=0.50\linewidth]{myPhoto/sixth.jpg}
    \caption{Создание множества с нулевой мощностью}
    \label{fig:edge2}
\end{figure}

Программа корректно обрабатывает граничные случаи, создавая пустые множества при необходимости и позволяя продолжить работу.