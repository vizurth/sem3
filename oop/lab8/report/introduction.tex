{
\section{Введение}

\par Современные информационные системы требуют эффективных инструментов для управления контактными данными. Телефонный справочник является базовым, но важным приложением, демонстрирующим принципы работы с данными, пользовательским интерфейсом и файловым хранилищем. В рамках данной лабораторной работы разработано консольное или десктопное приложение для управления телефонными контактами.

\subsection{Используемые технологии}

В ходе работы применяются следующие технологии и инструменты:

\begin{itemize}
    \item \textbf{Язык программирования:} C++ (стандарт C++17)
    \item \textbf{Фреймворк разработки:} Qt Framework (версия 6.10)
    \item \textbf{Среда разработки:} Visual Studio Code
    \item \textbf{Система сборки:} CMake
\end{itemize}

\subsection{Описание фреймворка Qt}

Qt представляет собой кросс-платформенный фреймворк для разработки приложений с графическим интерфейсом пользователя. Фреймворк Qt обладает следующими ключевыми особенностями:

\subsubsection{Механизм сигналов и слотов}

Основой архитектуры Qt является механизм сигналов и слотов, который обеспечивает гибкую систему обмена сообщениями между объектами. Сигнал излучается объектом при наступлении определённого события, а слот представляет собой функцию, вызываемую в ответ на этот сигнал.

\subsubsection{Виджеты и компоненты интерфейса}

Qt предоставляет обширную библиотеку готовых виджетов для построения графического интерфейса:
\begin{itemize}
    \item Контейнеры: QWidget, QMainWindow, QDialog
    \item Элементы ввода: QLineEdit, QTextEdit, QComboBox
    \item Кнопки: QPushButton, QRadioButton, QCheckBox
    \item Отображение данных: QTableWidget, QListWidget
    \item Диалоги: QMessageBox, QFileDialog, QInputDialog
\end{itemize}

\subsubsection{Работа с данными}

Qt включает мощные инструменты для работы с различными типами данных:
\begin{itemize}
    \item Работа с файловой системой через QFile, QDir
    \item Поддержка XML (QXmlStreamReader/Writer) и JSON (QJsonDocument)
    \item Возможности работы с базами данных SQL через Qt SQL модуль
    \item Контейнеры данных: QList, QVector, QMap, QString
\end{itemize}

\subsection{Предметная область — телефонный справочник}

Телефонный справочник — это приложение для хранения, организации и управления контактной информацией. Типичная запись справочника содержит:

\begin{itemize}
    \item \textbf{Личные данные:} фамилия, имя, отчество (опционально)
    \item \textbf{Контактная информация:} один или несколько номеров телефона
    \item \textbf{Дополнительные данные:} адрес электронной почты, физический адрес, примечания
\end{itemize}

Современный телефонный справочник должен обеспечивать следующие возможности:

\begin{enumerate}
    \item \textbf{Управление записями:}
    \begin{itemize}
        \item Добавление новых контактов с валидацией введённых данных
        \item Просмотр полного списка сохранённых контактов
        \item Редактирование существующих записей
        \item Удаление контактов
    \end{itemize}
    
    \item \textbf{Поиск и фильтрация:}
    \begin{itemize}
        \item Быстрый поиск по различным критериям (имя, фамилия, телефон, Email, дата рождения)
        \item Поддержка частичного совпадения при поиске
    \end{itemize}
    
    \item \textbf{Сортировка данных:}
    \begin{itemize}
        \item Упорядочивание по алфавиту
        \item Сортировка по дате добавления записи
    \end{itemize}
    
    \item \textbf{Персистентность данных:}
    \begin{itemize}
        \item Сохранение данных между сеансами работы приложения
        \item Автоматическое или ручное сохранение изменений
        \item Возможность экспорта и импорта данных
    \end{itemize}
\end{enumerate}

\newpage
}