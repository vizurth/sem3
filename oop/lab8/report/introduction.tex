{
\section*{Введение}
\addcontentsline{toc}{section}{Введение}

\par В рамках практики были изучены основные технологии разработки в операционной системе GNU/Linux. Освоены навыки работы с виртуальными машинами, командной строкой, системами сборки, а также создание и использование статических и динамических библиотек. В качестве среды использовалась минимальная установка Debian 12 в VirtualBox без графического интерфейса, что обеспечило полноценную работу в терминале.

\par Был проведён анализ компиляции программ с различными уровнями оптимизации, а также сравнительное исследование производительности рекурсивных и итеративных реализаций алгоритма возведения в степень целых чисел.

\par Целью работы стало понимание принципов функционирования операционной системы, взаимодействия с файловой системой и освоение методов работы с библиотеками. Практическая часть включала настройку доступа к виртуальной машине, выполнение команд в терминале, создание библиотек и разработку программ, способных обрабатывать бинарные файлы и использовать элементы псевдографики.

\par В рамках проекта была реализована программа с использованием библиотеки \textbf{ncurses}, предназначенная для работы с базой данных. Основное внимание уделялось обработке бинарных файлов и работе со структурами данных. Полученные знания и навыки являются актуальными для системного программирования и разработки кросс-платформенных приложений на базе Linux.


\newpage