{
\section{Постановка задач}
\par Цель исследования — освоение принципов функционирования ОС, взаимодействия с файловой системой, а также изучение принципов работы со статическими и динамическими библиотеками. Практическая часть включает в себя настройку доступа к виртуальной машине, выполнение команд через интерпретатор, создание библиотек и разработку программ, обрабатывающих бинарные файлы и использующих псевдографику. \\

\subsection{Установка и найстройка среды}
\par В рамках практики необходимо:
\begin{itemize}
    \item установить VirtualBox и загрузить в нём Debian 12 без графической оболочки;
    \item создать пользователя с логином своей фамилии и персональным именем хоста;
    \item настроить проброс порта 22 из гостевой ОС в хостовую для подключения по SSH.
\end{itemize}

\subsection{Основы работы в командной строке}
\par Следует освоить базовые приёмы работы в терминале \textbf{Linux}:
\begin{itemize}
    \item изучить ключевые команды;
    \item научиться пользоваться историей команд;
    \item настроить вход по SSH с использованием ключей.
\end{itemize}

\subsection{Подготовка среды разработки}
\par Для выполнения заданий нужно установить компилятор и инструменты разработки, включая \textbf{CMake}.
\subsubsection{Задание 1. Статические и динамические библиотеки}
\par Требуется:
\begin{enumerate}
    \item Реализовать функцию возведения в степень целых чисел:
    \begin{itemize}
        \item с использованием рекурсии;
        \item с применением итеративного подхода (без рекурсии).
    \end{itemize}
    \item Создать две версии библиотеки:
    \begin{itemize}
        \item статическую;
        \item динамическую (разделяемую).
    \end{itemize}
    \item Провести тестирование производительности с использованием различных флагов оптимизации компилятора:
    \begin{itemize}
        \item \texttt{-O0}, \texttt{-O1}, \texttt{-O2}, \texttt{-Os}.
    \end{itemize}
    \item Реализовать загрузку динамической библиотеки во время выполнения с использованием функций \texttt{dlopen}.
\end{enumerate}

\subsubsection{Задание 2. Двоичные файлы и визуализация структур в псевдографике}
\par Требуется:
\begin{enumerate}
\item В рамках индивидуального задания необходимо реализовать структуру с именем \textbf{ZNAK}, включающую следующие поля:
    \begin{itemize}
        \item фамилия (строка Си);
        \item имя (строка Си);
        \item знак Зодиака (перечисляемый тип).
        \item дата рождения (битовая структура).
    \end{itemize}
\item Требуется разработать дополнительную структуру, представляющую заголовок базы данных. Поля этой структуры:
    \begin{itemize}
        \item сигнатура формата файла (4 байта) — первые четыре буквы фамилии разработчика, записанные латиницей;
        \item номер транзакции (4 байта) — счётчик, увеличивающийся при каждом обращении к базе (чтение или запись);
        \item число записей (4 байта) — количество структур типа \textbf{ZNAK}, следующих за заголовком;
        \item контрольная сумма CRC-32 (4 байта) — вычисляется на основе всех байтов структур \textbf{ZNAK} с помощью функции из библиотеки \textbf{zlib.h}.
    \end{itemize}
\item Программа должна иметь псевдографический интерфейс, реализованный с использованием библиотеки \textbf{ncurses}. Реализуются следующие режимы работы:
    \begin{itemize}
        \item создание базы данных при её отсутствии:
            \begin{itemize}
                \item ввод массива структур \textbf{ZNAK} с клавиатуры;
                \item формирование бинарного файла и запись корректного заголовка.
            \end{itemize}
        \item Индивидуальное задание (поиск информации по фамилии людей):
            \begin{itemize}
                \item найти и вывести информацию о людях, чья фамилия введена с клавиатуры;
                \item при отсутствии соответствующего пользователя — вывести сообщение об этом;
                \item произвести обновление заголовка базы.
            \end{itemize}
        \item Добавление новой записи:
            \begin{itemize}
                \item ввод одной структуры \textbf{ZNAK};
                \item дозапись в файл и обновление заголовка (счётчик, количество, CRC).
            \end{itemize}
    \end{itemize}
\item После компиляции и запуска программы необходимо проанализировать подключённые разделяемые библиотеки с помощью утилиты \textbf{ldd}, а также построить рекурсивное дерево зависимостей.
\end{enumerate}

\newpage
}