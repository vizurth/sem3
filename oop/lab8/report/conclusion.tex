\section*{Заключение}

В ходе выполнения лабораторной работы была последовательно проведена всесторонняя оценка различных аспектов сборки и функционирования программного обеспечения, связанного с записью данных о людях и алгоритмами возведения в степень целого числа. Исследование охватило несколько ключевых направлений, каждое из которых дополняло общее понимание влияния архитектурных решений на производительность, структуру зависимостей и удобство эксплуатации программного продукта.

\textbf{1.1} Была успешно выполнена установка VirtualBox и установка Debian 12 без графической оболочки. Создан пользователь с логином на основе фамилии, настроен проброс порта 22 и выполнено подключение к системе по SSH.

\textbf{1.2} Освоены ключевые команды терминала, работа с историей команд, а также настройка входа по SSH с использованием ключевой аутентификации. Были выполнены типовые команды и проверены man-страницы с пояснением специфики.

Произведена установка компилятора, утилит make и cmake. Среда разработки подготовлена для выполнения практических заданий.

\textbf{1.3.1} Реализованы две версии алгоритма возведения в степень целого числа: итеративная и рекурсивная. Созданы статическая и динамическая версии библиотеки libpower. Для обеих реализаций проведено сравнительное тестирование производительности на множестве повторов с флагами компиляции -O0, -O1, -O2 и -Os. Зафиксировано, что итеративная версия демонстрировала наименьшее время выполнения при разных уровнях оптимизации.

Также реализована загрузка динамической библиотеки во время выполнения с помощью dlopen().

\textbf{1.3.2} Создана структура ZNAK, включающая поля для фамилии, имени, даты рождения и знака зодиака. Разработана структура заголовка \\ DB\_HEADER, содержащая сигнатуру, номер транзакции, количество записей и CRC32. Реализована система взаимодействия с бинарным файлом \\ znak.db: создание, дозапись и обновление заголовка.

Создан пользовательский интерфейс на базе ncurses с функциями: добавления записи, отображения списка записей, поиск записи по фамилии человека

Проанализированы зависимости каждого исполняемого файла с помощью утилиты ldd, составлено их рекурсивное дерево.

Таким образом, поставленные цели были достигнуты, а полученные результаты могут служить базой для построения более сложных систем с учётом требований к производительности, модульности и сопровождаемости.

\newpage


\begin{thebibliography}{9}
\bibitem{debian}
Debian Documentation. Официальный сайт Debian. \\
URL: \texttt{https://www.debian.org/doc/} (дата обращения: 20.08.2025)

\bibitem{cmake}
CMake Reference Documentation. Официальный сайт CMake. \\
URL: \texttt{https://cmake.org/documentation/} (дата обращения: 21.08.2025)

\bibitem{gcc}
Oracle VirtualBox. \\
URL: \texttt{https://www.virtualbox.org/} (дата обращения: 20.09.2025)

\bibitem{ncurses}
Новые проклятия: руководство по ncurses / Хабр. \\
URL: \texttt{https://habr.com/ru/articles/778040/} (дата обращения: 22.08.2025)
\end{thebibliography}
